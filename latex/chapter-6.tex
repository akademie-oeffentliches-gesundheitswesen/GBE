\documentclass{article}

                
\usepackage[backend=biber,hyperref=false,citestyle=authoryear,bibstyle=authoryear]{biblatex}
                
\bibliography{bibliography}
            
\begin{document}

\title{GBE06-Vernetzung}

\maketitle


\subsection{Vernetzung der verschiedenen Ebenen von Gesundheitsberichterstattung}\label{H2151291}



Die Gesundheitsberichterstattung auf kommunaler, Länder- und Bundesebene haben unterschiedliche Schwerpunkte, jedoch steht bei allen die Information zur Verbesserung der Gesundheit im Vordergrund. Daher ist es wichtig, dass die unterschiedlichen Ebenen sowohl horizontal  (innerhalb der Kommune und zwischen Kommunen) als auch vertikal (wie die Zusammenarbeit von Länder- und Bundes-GBE) zusammenarbeiten. Zu dieser Zusammenarbeit gehört vor allem die Harmonisierung der Datenquellen, von Indikatoren und gemeinsame Definition von Verfahren. Daneben ist eine gemeinsame thematische Schwerpunktsetzung oder die Zusammenarbeit bei der Berichterstellung denkbar.


\subsection{GBE als Teil eines Netzwerkes}\label{H5247316}



Gesundheitsberichterstatter\_innen bedürfen einer Reihe an methodisch-fachlichen Kompetenzen (Verweis Abschnitt Qualifikation/Handwerk), gleichzeitig agieren sie nicht in einem Vakuum. Berichterstattung ist bestenfalls ein interdisziplinärer, multiprofessioneller Prozess (Verweis integrierte Berichterstattung). Eine wesentliche Qualifikation der Berichterstatter\_innen neben methodisch-fachlichen Kompetenzen ist die Kommunikations- und Netzwerkkompetenz. Ihre Aufgabe ist verbunden mit anderen Bereichen der Verwaltung und Akteur\_innen aus den unterschiedlichsten Feldern, von der Arbeitsagentur bis zur Zahnprophylaxe. 


Gesundheitsberichterstattung steht nicht für sich alleine, sondern ist eingebettet in einen kontinuierlichen Prozess aus Problemdefinition, strategischer Planung, Umsetzung und Bewertung (Verweis Abschnitt Public Health Action Cycle). Weiter speist sich die Gesundheitsberichterstattung aus verschiedenen Themen, nicht nur rein medizinischen, sondern auch Themen mit Bezug zu den Lebensverhältnissen (z.B. Einkommen, Kultur, Bildung), zur Umwelt (z.B. Lärm, Schadstoffe, Hitze) oder zum Wohnen (z.B. Grünflächen, Infrastruktur) (Verweis Determinanten der Gesundheit). Während Berichterstatter\_innen ihre methodisch-fachlichen Kompetenzen bei der Berichterstellung einbringen, bedarf es ihrer kommunikativen Kompetenz, Partner\_innen für diesen Prozess zu gewinnen und Mut, sich gewinnen zu lassen. Idealerweise gibt es in der Kommune bereits eine Vernetzungsstruktur, auf die sie zurückgreifen können, z. B. Integrierte Gesundheits- und Sozialberichterstattung (Berlin), Sozialmonitoring (Stuttgart) oder Kommunale Gesundheitskonferenzen (KGK), die in verschiedenen Bundesländern gesetzlich verankert und etabliert sind (Verweis gesetzliche Grundlagen und Strukturen). 


\textbf{Mögliche Partner\_innen innerhalb der Verwaltung sind u.a.:}

\begin{itemize}
\item je nach Thema Kolleg\_innen aus dem eigenen Gesundheitsamt (z.B. Psychiatrie- und Suchtkoordination, Kinder- und Jugendgesundheitsdienst, Gesundheitsförderung, Gesundheitsplanung, Infektionsschutz, Umweltmedizin)


\item Statistikamt


\item Sozialamt


\item Jugendamt


\item Umweltamt


\item Katasteramt 


\end{itemize}
\begin{itemize}
\item Ämter für Stadtplanung und -entwicklung


\end{itemize}
\begin{itemize}
\item Schulverwaltungsamt / Amt für Bildung


\item Amt für Sport/Bewegung, Stadtsportbund


\item Beauftragte der Kommune für Kinder


\item Beauftragte der Kommune für die Belange von Menschen mit Behinderung 


\item Senior\_innen-Beauftragte / Senior\_innen-Beirat


\item Beauftragte der Kommune für Integration


\item Gleichstellungsbeauftragte


\item Pressestelle


\textbf{Mögliche Partner\_innen außerhalb der Verwaltung sind u.a.:}


\item ambulante und stationäre Versorgung


\item Rettungsdienste


\item ambulante und stationäre Pflege


\item niedergelassene Ärzt\_innen unterschiedlicher Fachrichtungen


\item Organe der Selbstverwaltung (Kassen(zahn)ärztliche Vereinigung, (Zahn-) Ärztekammer, Psychotherapeutenkammer)


\item gesetzliche und private Krankenkassen


\item Flüchtlingsrat


\item Medien


\end{itemize}
\begin{itemize}
\item Jobcenter/Agentur für Arbeit…


\item bereits vorhandene Arbeitsgruppen und Arbeitskreise, z.B. Suchtprävention, Psychosoziale Arbeitsgemeinschaft (PSAG)


\end{itemize}

\subsection{Themenbezogene Projektgruppe}\label{H1702121}



Für die Erstellung eines thematisch eingegrenzten Berichts bietet es sich gegebenenfalls an, eine temporäre Projektgruppe zum Berichtsthema zu gründen. Potenzielle Teilnehmer\_innen der Projektgruppe sind alle Personen in der Kommune, die etwas zum Thema beitragen können und die möglicherweise bei der Umsetzung von Maßnahmen, die aus dem Bericht folgen könnten, betroffen sind. Dies könnten zum Beispiel Träger von Einrichtungen der Kinder- und Jugendhilfe oder Einrichtungen der Altenhilfe sein.


Erfahrungsgemäß gibt es häufiger Vorbehalte von einzelnen Institutionen, Ressorts, Ämtern oder Interessensvertreter\_innen gegenüber einem geplanten Gesundheitsbericht. Sei es, dass sie fürchten, in einem Bericht nicht gut "dazustehen" oder dass sie Sorge haben, sich den Handlungsempfehlungen nicht gewachsen zu fühlen mangels Ressourcen und/oder politischen Rückhalts der Verwaltung und Kommunalpolitik etc. Hier ist es sinnvoll, insbesondere diese Gruppen von Anfang an einzubeziehen und das Vorgehen (Datenerhebung, Auswertung, Interpretation) transparent zu machen. Gleichzeitig müssen Berichterstatter\_innen kommunikatives Geschick haben, um sich aktiv in Prozesse einzubringen und von anderen als Partner\_innen mit starker Stimme für gesundheitsorientierte Themen wahrgenommen zu werden.


Insbesondere bei der Interpretation der Ergebnisse und der daraus abzuleitenden Handlungsempfehlungen kommt der Projektgruppe bzw. den betroffenen Institutionen eine wichtige Rolle zu. Dabei haben Gesundheitsberichterstatter\_innen die Rolle, gemeinsam mit Partner\_innen/Akteur\_innen wissenschaftlich fundierte Handlungsempfehlungen zu erarbeiten, wobei sie die Umsetzbarkeit vor Ort im Blick haben müssen (Verweis Planung, Lit. Fachplan Gesundheit). Hier bewegen sich Gesundheitsberichterstatter\_innen im Spannungsfeld Wissenschaft, Praxis und - nicht zuletzt - Politik.


Insbesondere bei der Integration verschiedener Berichtserstattungssysteme (Verweis Abschnitt integrierte Berichterstattung) werden unterschiedliche Schnittstellen in Anspruch genommen, und neben dem fachlichen Austausch steht die Klärung und Integration unterschiedlicher Erwartungen und Interessen an. Aus Sicht der beteiligten Ressorts kann es zu unterschiedlichen Prioritäten kommen, ebenso wie die Gesundheitsberichterstattung als reine Dienstleisterin für andere Ressorts betrachtet werden kann, was ihrer Stellung innerhalb des Systems nicht gerecht wird.

Bei der (technischen) Integration verschiedener Berichterstattungssysteme innerhalb einer Kommune kann es bezüglich der räumlichen Bezugsebene für die Gesundheitsdaten zu Schwierigkeiten kommen. Sozial- oder Einwohnerdaten können sehr kleinräumig dargestellt werden (Stadtteile, Quartiere, Baublocks). Für Gesundheitsdaten ist dies aus Gründen des Datenschutzes oft nicht möglich, weil die Fallzahlen zu gering sind. Deshalb müssen hier gemeinschaftlich Lösungen für eine einheitliche Darstellung gefunden werden (Verweis auf Kapitel Datenschutz) bzw. ein Konzept entwicklet werden, in dem sich Daten auf unterschiedlichen räumlichen Ebenen sinnvoll ergänzen. Die Lösung dieses Problems kann bspw. darin bestehen, Sozialdaten und Daten aus weiteren Ressorts kleinräumig auszuwerten, umbesonders belastete Sozialräume zu identifizieren. Da Zusammenhänge zur gesundheitlichen Lage aus epidemiologischen Studien hinreichend nachgewiesen sind, kann der Bedarf für Gesundheitsförderungsmaßnahmen daraus abgeleitet werden. Darüber hinaus können Gesundheitsdaten auf der Ebene eines Verwaltungsbezirks (Kreis oder kreisfreie Stadt) Hinweise auf besondere gesundheitliche Belastungen in dem jeweiligen Kreis/der kreisfreien Stadt liefern \autocite{RosenkötterNicoleundweitere2020}.


\subsection{Politik und Entscheidungsträger\_innen }\label{H3426797}



Egal ob der Auftrag für einen Gesundheitsbericht "von oben" erteilt wurde oder ob die Notwendigkeit für einen Bericht aus der GBE selbst kam: wichtige Entscheidungsträger\_innen sollten kontinuierlich informiert, gegebenenfalls überzeugt und eingebunden werden (siehe Strukturen). 


Dazu gehören:

\begin{itemize}
\item Amts- und Abteilungsleitung


\item Dezernent\_in 


\item Landrat oder Landrätin, (Ober-)Bürgermeister\_in


\item Gremien wie Kreistag, Stadtrat, Gemeinderat, Ortschaftsrat


\end{itemize}

Die Einbindung der politischen Leitungsebene ist ohnehin notwendig, weil ein Gesundheitsbericht in der Regel nicht vom GBE-Team veröffentlicht wird, sondern vom Landratsamt oder der Stadt. Die politische Leitung muss den Bericht also vertreten können. Hinzu kommt, dass Partner\_innen für die Verbreitung eines Berichts gebraucht werden. Die Wahrnehmung der Berichterstattung hängt ganz entscheidend davon ab, wie gut vorher der Weg bereitet wurde. Im schlimmsten Fall kann es passieren, dass ein Bericht in der Schublade verschwindet, oder sich niemand imstande sieht, sich der Themen anzunehmen bzw. Verantwortung zu übernehmen. Die zeitliche Komponente ist ebenfalls entscheidend: Sitzungstermine der Gremien, Haushaltsberatungen, Kommunalwahlen, Ausnahmesituationen (Corona, Migrationsdynamik), Sommerpause oder "Sommerloch". Neben dem Interesse spielen dabei auch die Kompetenzen der Adressat\_innen eine wesentliche Rolle.


\subsection{Capacity Building oder Kompetenzentwicklung innerhalb des Netzwerkes}\label{H3492218}



Im Vorangegangen ist die Kompetenzentwicklung von Gesundheitsberichterstatter\_innen bereits beschrieben worden (Verweis Kapitel Handwerkszeug/Qualifikation). Dies ist die Grundlage für Berichterstatter\_innen, die sie in die Lage versetzt, den Aspekt der Vernetzung zu verfolgen. Innerhalb des Netzwerkes ist die Kompetenzentwicklung eine fortlaufender Lernprozess aller Beteiligten. Capacity Building (Kompetenzentwicklung) beschreibt ein prozesshaftes Geschehen, bei dem das voneinander Lernen im Mittelpunkt steht. Hier hilft es Berichterstatter\_innen, "einen langen Atem zu haben" und Frustrationserlebnisse als Teil dieses Prozesses einzuordnen und nicht den Mut zu verlieren.


Die Kooperation und das vernetzte Arbeiten über verwaltungsinterne und externe Ressorts und Sektoren hinweg erfordert ein stetiges Bewusstsein eigener Positionen und Interessen und der der anderen Partner\_innen (Lit. Netzwerkarbeit, Harvard-Konzept).


Gesundheitsberichterstattung erfordert auf Seiten der Berichterstatter\_innen verschiedene Qualifikationen (siehe oben und Verweis Abschnitt Qualifikation/Qualifzierung GBEler\_innen), gleichzeitig sind für einen verantwortungsvollen Umgang mit Gesundheitsberichten gewisse Kompetenzen nötig und erwünscht. Gesundheitsberichterstattung steht im Spannungsfeld von Wissenschaft, Politik, Medien und Öffentlichkeit, bestenfalls ist sie Mittlerin zwischen wissenschaftlichen Erkenntnissen bzw. Ergebnissen empirisch fundierter Analysen und politisch Handelnden bzw. Bürger\_innen (siehe Selbstverständnis, Kap. 2). - Brücke: Wiss. + GBE Praxis Sie beschreibt die gesundheitliche Lage der Bevölkerung mittels unterschiedlicher Daten und Kennzahlen, verknüpft diese mit wissenschaftlichen Erkenntnissen zu gesundheitsbezogenen Themen, interpretiert und formuliert, sinnvollerweise gemeinsam mit anderen Akteur\_innen und Expert\_innen,  Handlungsempfehlungen. Mit Ausnahme der Fachöffentlichkeit und Wissenschaftsjournalist\_innen sind Empfänger\_innen der in Gesundheitsberichten dargestellten Inhalte mehrheitlich weder im Umgang mit Daten noch mit wissenschaftlichen Aussagen geschult. Daraus resultiert auf der einen Seite der Anspruch an die Gesundheitsberichterstattung, Adressat\_innengerecht zu formulieren und Sachverhalte darzustellen (Verweis Abschnitt Sprache/Begrifflichkeiten). Auf der anderen Seite ist eine Kompetenzentwicklung im Umgang mit Gesundheitsberichten erstrebenswert.


Kompetenzentwicklung bei Adressat\_innen von Gesundheitsberichten zielt auf unterschiedliche Bereiche ab. Hier sind zu nennen: Umgang mit Daten, Grafiken, gesundheitsbezogenen Themen und darauf basierenden Empfehlungen. 

Nun stellen Daten/Zahlen im Allgemeinen für zahlreiche Menschen eine Herausforderung dar (Lit. Kuhn, Gesundheitsdaten verstehen). Sich diesen anzunähern und ein Verständnis dafür zu entwickeln – ohne die Tiefen der Statistik zu durchdringen – erfordert Offenheit und Respekt für das Dargestellte. Es bedarf einer unvoreingenommenen Haltung, die durch die oben beschriebene Einbindung in den Prozess der Berichterstattung erleichtert werden kann. In der Berichterstattung fehlt es oft an Möglichkeiten Ursache-Wirkungs-Mechanismen durch Daten zu belegen. Sie bewegt sich häufig auf der Ebene, unterschiedliche Beobachtungen miteinander in Verbindung zu bringen (Assoziationen) ohne über miteinander verbundene Daten zu verfügen (Beispiel: Verknüpfung prozentualem Anteil Adipositas mit prozentualem Anteil nicht-autochtoner Menschen). Um solchen ökologischen Fehlschlüssen vorzubeugen, ist nicht nur Obacht bei der Berichterstattung von Nöten, sondern eine vorurteilsfreie Auseinandersetzung mit dem Bericht selbst auf Seiten der Adressat\_innen.

Grafiken – und insbesondere kartographische Darstellungen – sind beliebte Elemente in der GBE, die Sachverhalte veranschaulichen können (Verweis Abschnitt Erstellung von Grafiken, Lit. GkPiG 2016). Gleichermaßen bergen sie unzählige Möglichkeiten zur Manipulation. Dass dies nicht opportun und wider den Prinzipien wissenschaftlichen Arbeitens ist (Verweis Abschnitt Prinzipien wissenschaftlichen Arbeitens), wird von der Boulevardpresse oftmals (bewusst) ignoriert. Berichterstatter\_innen sollten deshalb kompetent sein, Grafiken zu erstellen. Gleichwohl können sie nicht intendierten Fehlinterpretationen den Weg ebnen, wenn mit der Absicht, Darstellungen zu vereinfachen, ein Format gewählt wird, dass etwa Unterschiede überbetont (Beispiel: x-Achse schneidet y-Achse nicht bei Null). Adressat\_innen der GBE bedürfen der Kompetenz Grafiken zu "lesen" und zu interpretieren. Hier kann es von Nutzen sein, unterschiedliche Darstellungen ein und desselben Sachverhalts exemplarisch und außerhalb des Berichts zu präsentieren, um ein Verständnis für die Vielfalt der Darstellungsoptionen zu entwickeln (Beispiel Oberwöhrmann Grafiken? -- klären!). 

Die inhaltliche Auseinandersetzung mit gesundheitsbezogenen Themen unter Berücksichtigung wissenschaftlicher Erkenntnisse stellt für Berichterstatter\_innen wie Rezipient\_innen eine Herausforderung dar. Die günstigste Konstellation (abgesehen von Sachverständnis auf beiden Seiten) ist diejenige, bei der Berichterstatter\_innen ein Thema so durchdringen, dass sie komplexe Inhalte einfach, ohne banal zu wirken, beschreiben können. Die Fähigkeit sich unvoreingenommen auf unbekannte Sachverhalte einzulassen bedarf bei den Adressat\_innen des Berichts einer großen Portion Neugier. Medizinische, gesellschaftliche und psychologische Einflussfaktoren auf Gesundheit (Verweis Abschnitt Determinanten) sind oftmals in der Bevölkerung wenig bekannt, sodass die Auseinandersetzung damit schnell zu einer individuellen Zuschreibung der Verantwortung führt statt strukturell bedingte Vulnerabilität in den Blick zu nehmen.

Wie oben bereits angesprochen, stehen die aus den Ergebnissen des Berichts abgeleiteten Handlungsempfehlungen unter "besonderer Beobachtung". Selbst wenn die Formulierung wissenschaftlich fundierter Handlungsempfehlungen durch die Berichterstatter\_innen gemeinsam mit Expert\_innen erfolgt, besteht die Gefahr, dass Adressat\_innen sich diesen nicht gewachsen fühlen, sie nicht als ihr originäres Handlungsfeld ansehen oder sie schlicht ablehnen. Handlungsempfehlungen müssen auf Resonanz treffen um eine Chance auf Umsetzung zu haben. Ähnlich wie ein Cello einen wunderbaren Resonanzkörper hat, der aber ohne sachkundige Cellospielerin oder sachkundigen Cellospieler nie klingen wird, werden Berichte wenig Wirkung entfalten, wenn Handlungsempfehlungen nicht auf Resonanzfähigkeit seitens der Akteur\_innen treffen.





\subsubsection{}\label{H9112914}



\printbibliography[title={Literaturverzeichnis}]
\end{document}
