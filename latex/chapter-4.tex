\documentclass{article}

\usepackage{hyperref}
                
\usepackage[backend=biber,hyperref=false,citestyle=authoryear,bibstyle=authoryear]{biblatex}
                
\bibliography{bibliography}
            
\usepackage{tabu}
\usepackage{multirow}
\begin{document}

\title{GBE04-GBE Handwerk}

\maketitle





\subsubsection{\textbf{Datengrundlagen der Gesundheitsberichterstattung}}\label{H2929607}



Gesundheitsberichterstattung arbeitet zumeist mit so genannten Sekundärdaten. Als Sekundärdaten bezeichnet man Daten, die in anderen Zusammenhängen entstanden bzw. die für andere Zwecke erhoben worden sind. Beispiele sind amtliche Statistiken wie Bevölkerungsstatistiken, die Todesursachenstatistik oder die obligatorischen Statistiken der Gesetzlichen Krankenversicherung (KV 45, KJ 1, KM 1) und der Krankenhäuser. Zu den für die GBE nutzbaren Sekundärdaten zählen auch Registerdaten (Krebsregister, Herzinfarktregister) sowie Routinedaten und prozessgenerierte Daten. Routinedaten bzw. prozessgenerierte Daten sind Informationen, die Behörden, Organisationen und Unternehmen im Zuge ihrer Aktivitäten erheben. Bekannte Beispiele sind die Daten aus den Schuleingangsuntersuchungen der Gesundheitsämter oder die gemäß dem Infektionsschutzgesetz (IfSG) erhobenen Informationen. Auch die Daten der Sozialversicherungen und der Kassenärztlichen Vereinigungen können für die Gesundheitsberichterstattung von Bedeutung sein. 


 Weitere Datenquellen für die Gesundheitsberichterstattung sind wissenschaftliche Studien wie die Studie zur Gesundheit von Kindern und Jugendlichen in Deutschland (KiGGS), die häufig als Public-Use-Files zur Verfügung stehen. Public-Use-Files können für eigene statistische Auswertungen genutzt werden, um zu neuen, für den Berichtsgegenstand relevanten Erkenntnissen zu gelangen. 


Für die Nutzung von Sekundärdaten sprechen in erster Linie forschungsökonomische Überlegungen. Sekundärdaten müssen nicht erst mühsam erhoben werden, sondern liegen schon aufbereitet in auswertbarer Form vor. Idealerweise wurden sie standardisiert erfasst, man kann also in der Regel eine gewisse Datenqualität unterstellen. Dies gilt insbesondere für Survey-Daten.


Tab. in den Anhang


\begin{tabu} to \textwidth { |X|X|X|X|X| }
\hline



\textbf{Themenbereich} & \textbf{Datenquellen}
 & \multicolumn{2}{c}{

\textbf{Indikatorenbeispiele}
} & 

\textbf{abrufbar bei}
 \\
\multirow{10}{*}{Daten zur gesundheitlichen Lage} & 

Todesursachenstatistik, Bevölkerungsstatistik, 

Sterbetafeln, Geburtenstatistik…
 & \multicolumn{2}{c}{

Sterblichkeit (Mortalität), Haupttodesursachen 

vermeidbare Sterblichkeit

Müttersterblichkeit/Säuglingssterblichkeit

Lebenserwartung

Verlorene Lebensjahre (PYLL)...
} & 

Statistische (Landes-)Ämter
 \\


Krankenhausstatistik
 & \multicolumn{2}{c}{

Krankenhausbehandlungsraten (Morbidität)

personelle und sachliche Ausstattung

Kostennachweis der Krankenhäuser...
} & 

www.gbe-bund.de, www.destatis.de
 \\


Reha-Statistik DRV
 & \multicolumn{2}{c}{

Leistungen zur Rehabilitation 

(Anträge, Leistungen etc.)...
} & 

www.destatis.de
 \\


Pflegestatistik
 & \multicolumn{2}{c}{

Anzahl Pflegebedürftiger Personen

Pflegeeinrichtungen und Plätze

Personal...
} & 

www.destatis.de
 \\


GKV-Routinedaten
 & \multicolumn{2}{c}{

stationäre Behandlungsfälle (Morbidität)

Diagnosen, Prozeduren

verordnete Arzneimittel

Angaben zur Arbeitsunfähigkeit 

Arbeitsunfähigkeitsdiagnosen /-Tage

Inanspruchnahme Vorsorgeuntersuchungen/ Früherkennungsuntersuchungen...
} & 

kommunal verfügbar?


in NRW nur AU-Daten der BKKen kommunal routinemäßig nutzbar
 \\


DRG-Statistik
 & \multicolumn{2}{c}{

abgerechnete Krankenhausbehandlungen (Diagnosen, Prozeduren etc.)...
} & 


 \\


Statistik der Straßenverkehrsunfälle
 & \multicolumn{2}{c}{

Beteiligte, Verletzte, Verkehrstote

Unfallursachen...
} & 

www.destatis.de
 \\


Daten der Kassenärztlichen Vereinigungen
 & \multicolumn{2}{c}{

ambulante Behandlung
} & 


 \\


Daten der Schuleingangsuntersuchung
 & \multicolumn{2}{c}{

Impfquoten, Teilnahmequote an U-Untersuchungen, Prävalenz verschiedener Befunde (z. B. Adipositas, hearbgesetzte Sehschärfe) und Auffälligkeiten des Entwicklungsstands (z. B. Körperkoordination), eigene Erhebungen
} & 

Kommunale Gesundheitsämter, Landesgesundheitsämter (aggregierte Daten auf Landesebene, z.T. auf Ebene der Kreise und kreisfreien Städte)
 \\


...
 & \multicolumn{2}{c}{

...
} & 


 \\
\multirow{2}{*}{Demographische Daten} & 

Fortschreibung des Bevölkerungsstandes (z.B. Stat. Jahrbuch)


Bevölkerungsstatistiken

(z.B. Statistische Ämter…)

…
 & \multicolumn{2}{c}{

Bevölkerungszahl

Geschlecht

Alter

Migration

Ausländische Bevölkerung

Bevölkerungsentwicklung…
} & 

Statistische (Landes-)Ämter
 \\


...
 & \multicolumn{2}{c}{

...
} & 


 \\


Daten zur sozialen Lage & Mikrozensus


Volkswirtschaftliche Gesamtrechnung der Länder...
 & \multicolumn{2}{c}{

Bildungsstand

Einkommen

Erwerbstätigkeit

Arbeitslosigkeit

Sozialhilfe…
} & 

www.destatis.de
 \\


Public use files & Gesundheitssurveys des Robert-Koch Instituts (DEGS, KiGGS, GEDA, )

Sozio-ökonomisches Panel (SOEP)

ALLsprävalenz ausgewählter Diagnosen &  &  & 
 \\


Daten der Schuleingangsuntersuchung
 & \multicolumn{2}{c}{

Impfquoten, Teilnahmequote an U-Untersuchungen, Prävalenz verschiedener Befunde (z. B. Adipositas, hearbgesetzte Sehschärfe) und Auffälligkeiten des Entwicklungsstands (z. B. Körperkoordination), eigene Erhebungen
} & 

Kommunale Gesundheitsämter, Landesgesundheitsämter (aggregierte Daten auf Landesebene, z.T. auf Ebene der Kreise und kreisfreien Städte) & 
 \\


...
 & \multicolumn{2}{c}{

...
} & 

 & 
 \\
\multirow{2}{*}{Demographische Daten} & 

Fortschreibung des Bevölkerungsstandes (z.B. Stat. Jahrbuch)


Bevölkerungsstatistiken

(z.B. Statistische Ämter…)

…
 & \multicolumn{2}{c}{

Bevölkerungszahl

Geschlecht

Alter

Migration

Ausländische Bevölkerung

Bevölkerungsentwicklung…
} & 

Statistische (Landes-)Ämter
 \\


...
 & \multicolumn{2}{c}{

...
} & 


 \\


Daten zur sozialen Lage & Mikrozensus


Volkswirtschaftliche Gesamtrechnung der Länder...
 & \multicolumn{2}{c}{

Bildungsstand

Einkommen

Erwerbstätigkeit

Arbeitslosigkeit

Sozialhilfe…
} & 

www.destatis.de
 \\


Public use files & Gesundheitssurveys des Robert-Koch Instituts (DEGS, KiGGS, GEDA, )

Sozio-ökonomisches Panel (SOEP)

ALLBUS 

…
 & \multicolumn{2}{c}{

…


kommunale Daten?


 
} & 

RKIich (Günter) ergänze die Tabelle im Nachgang
 \\
\hline

\end{tabu}

Eine wichtige Datenbank, in der viele Daten für Deutschland und die Bundesländer zu finden sind, ist die Datenbank der Bundesgesundheitsberichterstattung - online unter www.gbe-bund.de. Dort sind einzelne Sachverhalte über Suchbegriffe wie in einer Suchmaschine recherchierbar und die Tabellen oft nach den eigenen Bedürfnissen veränderbar. Allerdings sind dort bisher keine Daten auf Kreisebene abrufbar. Die muss man z.B. in den Sammlungen von Gesundheitsindikatoren der Länder suchen oder in den jeweiligen Originalquellen. Gesundheitsberichterstattung ist immer auch ein Suchen und Finden. 


Eine zweite wichtige Datenbank ist die Datenbank INKAR des Bundesinstituts für Bau-, Stadt- und Raumforschung - online unter www.inkar.de. Hier sind bis auf Kreisebene viele auch für die Gesundheitsberichterstattung relevante Daten abrufbar, vor allem zu sozialen, ökonomischen oder siedlungsstrukturellen Merkmalen. Es gibt auch einige Gesundheitsindikatoren, unter anderem sind hier Daten zur Lebenserwartung auf Kreisebene zu finden, die es in der amtlichen Statistik sonst nicht gibt.


\textbf{Datenerhebungen}


Für viele Berichtsthemen sind Sekundärdaten vollkommen ausreichend, manchmal jedoch decken sie nicht alle Aspekte eines Berichtsthemas ab. In solchen Fällen muss der/die Berichterstattende die fehlenden Informationen selbst erheben oder versuchen, neue Datenquellen zu akquirieren. 


Für eigene Erhebungen nutzt die Gesundheitsberichterstattung das methodische Instrumentarium der empirischen Sozialforschung. In aller Regel ist dies die standardisierte (quantitative) Befragung, andere Methoden der Datengewinnung wie die Beobachtung oder das Interpretieren von Texten (Inhaltsanalyse) sind ausgesprochen selten. Einen allgemein verständlichen Überblick über die Methoden der empirischen Sozialforschung gibt Diekmann (2007). 


In der empirischen Sozialforschung lassen sich zwei methodologisch unterschiedliche Positionen unterscheiden: Zum einen der quantitative Forschungsansatz, der sich an naturwissenschaftlicher Methodologie orientiert, zum anderen der interpretativ-qualitative Ansatz, dessen Wurzeln in der Ethnologie und in der frühen stadtsoziologischen Forschung liegen. 


In der Gesundheitsberichterstattung wie auch in der Gesundheitsforschung allgemein dominiert der quantitative Ansatz. Auswertungen der führenden epidemiologischen und medizinischen Zeitschriften ergaben, dass sich der Anteil der publizierten Studien, in denen qualitative Verfahren eine Rolle spielten, im unteren einstelligen Prozentbereich bewegt \emph{(Anmerkung: Quellen werden nachgereicht}). 


Seit einiger Zeit wird in der Gesundheitsberichterstattung verstärkt über die Nutzung unstandardisierter, qualitativer Verfahren diskutiert. Daher seien an dieser Stelle kurz die wesentlichen Unterschiede zwischen den beiden Forschungsansätzen beschrieben. 


Kennzeichnend für den quantitativen Ansatz ist die strikte Trennung zwischen Forschenden und Beforschten, der Einsatz standardisierter Erhebungsinstrumente in kontrollierten Erhebungssituationen sowie das Erreichen möglichst großer Fallzahlen, um innerhalb vertretbarer Fehlergrenzen durch statistische Auswertungen allgemeingültige Aussagen zu gewinnen. Um begründen zu können, warum man was erhebt, setzt dieser Ansatz allerdings ein Mindestmaß an Vorwissen über den Untersuchungsgegenstand voraus.


Demgegenüber versucht der qualitative Ansatz ausdrücklich, die subjektiven Sichtweisen der Beforschten kennenzulernen und daraus ableitbare Handlungen zu verstehen. Dies erfordert eine offene Herangehensweise mit möglichst wenig standardisierten Methoden wie Beobachtungen oder offene Interviews. Das dabei gewonnene umfangreiche Datenmaterial -  Videos, Beobachtungsprotokolle oder transkribierte Interviewtexte - wird anschließend aufwendig analysiert. Von daher basieren qualitative Studien höchstens auf einigen Dutzend Fällen. Wegen des offenen, teilweise  intuitiven Vorgehens und der geringen Fallzahlen wird die Validität der Ergebnisse qualitativer Studien jedoch häufig angezweifelt. 


Die jeweiligen Stärken beider Ansätze lassen sich innerhalb von Mixed Methods-Designs nutzen. Aufgrund ihrer Offenheit eignen sich qualitative Verfahren insbesondere zur Gewinnung neuer Erkenntnisse und zur Generierung von Hypothesen, die anschließend mithilfe quantitativer Verfahren geprüft werden können. U.a. wegen des damit verbundenen Aufwands ist dies im Forschungskontext häufiger als in der GBE gefragt.Datenquellen in der GBE


Einige wichtige Datenquellen der Gesundheitsberichterstattung werden im Folgenden kurz vorgestellt:


\textbf{Todesursachenstatistik}


Die Todesursachenstatistik ist eine der häufig genutzten Datenquellen der Gesundheitsberichterstattung. Sie gehört zur "amtlichen Statistik", also zu den durch Gesetz geregelten und den Statistischen Ämtern übertragenen Statistiken. Gesetzliche Grundlage der Todesursachenstatistik in Deutschland ist das Gesetz über die Statistik der Bevölkerungsbewegung und die Fortschreibung des Bevölkerungsstandes (Bevölkerungsstatistikgesetz). Die Todesursachenstatistik wird in Deutschland seit mehr als 100 Jahren erhoben und liegt nach Vorgaben der Weltgesundheitsorganisation in ähnlicher Form auch international vor. 


In der Todesursachenstatistik wird nach der Internationalen statistischen Klassifikation der Krankheiten (ICD) codiert (https://www.dimdi.de/dynamic/de/klassifikationen/icd/). Dabei wird in Deutschland derzeit noch eine monokausale Todesursachenstatistik geführt: In der Statistik wird das "Grundleiden" codiert, die Krankheit, die medizinisch ursächlich für den Tod ist. Das ist häufig nicht der unmittelbar den Tod auslösende Befund. Grundlage der Todesursachenstatistik sind die ärztlichen Eintragungen auf der Todesbescheinigung. Sie werden von den Standesämtern über die Gesundheitsämter an die Statistischen Landesämter übermittelt.


Es ist vorgesehen, auch in Deutschland eine multikausale Todesursachenstatistik einzuführen, in der alle Eintragungen auf der Todesbescheinigung zur Kausalkette der Todesursachen genutzt werden. Darüber hinaus gibt es das Bestreben, die Validität der Todesursachenstatistik weiter zu verbessern, sowie die Latenz zur Bereitstellung der Daten zu verkürzen \autocite{EckertOlafundweitere2018}.                     


Die Daten lassen sich nach Alter, Geschlecht, Staatsangehörigkeit und Wohnort differenzieren. Sie liegen auf Kreisebene und z.T. auch noch kleinräumiger vor. Eine für die Gesundheitsberichterstattung gravierende Einschränkung der Aussagekraft der Todesursachenstatistik besteht darin, dass sie keine Angaben zum Sozialstatus der Verstorbenen enthält. 


\textbf{Krankenhausstatistik}


Auch die Krankenhausstatistik ist eine amtliche Statistik. Rechtliche Grundlage ist hier die Krankenhausstatistik-Verordnung. Die Diagnosedaten sind wie bei der Todesursachenstatistik ICD-codiert und dokumentieren die Hauptbehandlungsdiagnose. Sie sind ebenfalls nach Alter und Geschlecht sowie nach Wohnort und Behandlungsort differenzierbar. Auch diese Daten liegen auf Kreisebene vor. In der Krankenhausdiagnosestatistik werden Behandlungsdiagnosen geführt. Das bedeutet, dass Mehrfachbehandlungen einer Diagnose eines Patienten mehrfach innerhalb eines Kalenderjahres gezählt werden können. 


Darüber hinaus liefert die Krankenhausstatistik wichtige Strukturdaten über die Krankenhäuser, z.B. über Bettenzahlen, Liegedauer oder das Personal der Krankenhäuser, dies jedoch meist nicht bis auf die Kreisebene herab.


Neben der amtlichen Krankenhausstatistik gibt es auch Krankenhausdaten aus der DRG-Statistik, also aus dem Abrechnungssystem der Krankenhäuser. Allerdings sind die DRG-Daten auf kommunaler Ebene nicht routinemäßig über die Statistischen Landesämter verfügbar.


\textbf{Schuleingangsuntersuchungen}


Bei der Berichterstattung zur Kindergesundheit sind die Daten der Schuleingangsuntersuchung eine wichtige Datenquelle. Sie werden von den Gesundheitsämtern selbst erhoben. Allerdings liegen sie in länderspezifischer Form vor, d.h. es gibt kaum zwischen den Bundesländern vergleichbare Daten aus den Schuleingangsuntersuchungen. Der Merkmalskatalog umfasst in der Regel Angaben zu Seh- und Hörfähigkeit, zu den motorischen Fähigkeiten der Kinder, ihrer geistigen Entwicklung, der Sprachfähigkeit sowie zu Gewicht und Größe. Auch der Impfstatus der Kinder wird in den Schuleingangsuntersuchungen dokumentiert. In manchen Bundesländern werden auch sozialstrukturelle Merkmale sowie Angaben zur Lebenssituation  und zum Migrationshintergrund erhoben. Anhand dieser Daten lässt sich eindrucksvoll die Abhängigkeit der Gesundheit von der sozialen Lage schon im Kindesalter zeigen. Im Allgemeinen gilt: Je niedriger der Sozialstatus der Familie, desto schlechter auch der gesundheitliche Entwicklungsstand des Kindes.


Die letzte Aussage gehört m.E. irgendwo spezifiziert


Die Daten der Schuleingangsuntersuchungen zeichnen sich dadurch aus, dass sie für eine wichtige biografische Phase der Kindheit einen vollständigen Jahrgang erfassen. Außerdem sind es Daten, die das Gesundheitsamt selbst erhebt, deren Aussagekraft es also auch gut beurteilen kann. WSchuleingangsdaten sind Screening-Daten, und sie können den aktuellen Gesundheitszustand nur ausschnittsweise beschreiben. Der psychische Gesundheitszustand wird z.B. allenfalls in der ärztlichen Anamnese erfasst, jedoch nicht systematisch erhoben.  Potenziell bieten die Schuleingangsuntersuchungen auch die Möglichkeit, für spezifische Fragestellungen zusätzliche (kurze) Frageinstrumente zu integrieren.


Die Informationen aus der Schuleingangsuntersuchung bieten ein gutes Potential für planungsrelevante Analysen unterhalb der Kreisebene und werden vielerorts z.B. im Rahmen einer integrierten Gesundheits- und Sozialberichterstattung dafür genutzt. Kleinräumige Auswertungen dürfen jedoch nur unter der Voraussetzung erfolgen, dass der Datenschutz der personenbezogenen Daten gewährleistet bleibt und die Kalibrierung der Untersucher\_innen ermöglicht, Untersucher\_inneneffekte methodisch sauber von räumlichen Effekten abzugrenzen.\textbf{Daten zu Infektionskrankheiten}


Eine zweite wichtige Datenquelle für die Gesundheitsberichterstattung, über die die Gesundheitsämter selbst verfügen, sind die Daten aus dem Meldewesen über Infektionskrankheiten. Das Infektionsschutzgesetz sieht vor, dass für bestimmte Infektionskrankheiten der Verdacht, der Laborbefund oder die Erkrankung meldepflichtig sind. Daran knüpfen sich in der Arbeit des Gesundheitsamtes weitreichende Folgen an, bei einem Masernausbruch in einer Kita beispielsweise Umgebungsuntersuchungen oder bei Infektionen in Lebensmittelbetrieben vielleicht ein Betätigungsverbot für die betroffenen Beschäftigten. Zudem eignen sich die Daten für manche Themen der Gesundheitsberichterstattung. So ist beispielsweise die Entwicklung der Masernzahlen, zusammen mit der Entwicklung der Masernimpfquoten (aus den Schuleingangsuntersuchungen) auch für die breitere Öffentlichkeit von Interesse.


Die Daten des Meldewesens unterliegen natürlich strengen Datenschutzanforderungen, es sollen ja nicht einzelne erkrankte Menschen identifiziert und dann womöglich stigmatisiert werden. Zudem ist zu bedenken, dass die Daten auf Meldungen vor allem von Ärzt\_innen und Laboren beruhen. Es sind keine Daten aus epidemiologischen Erhebungen, sie sind mit vielfältigen Selektionseffekten behaftet. Bei den Masern weiß man beispielsweise, dass die Meldedaten das reale Infektionsgeschehen deutlich unterschätzen. [Takla/Wichmann et al. measles incidence an reporting trends in Germany 2007 -2011, Bulletin of the Word Health Organizaton 2014]


\textbf{Bundesweite Gesundheitssurveys}


Eine wichtige Datengrundlage der Gesundheitsberichterstattung  besonders im Hinblick auf Gesundheitsverhalten und gesundheitliche Ungleichheit sind die Daten des Gesundheitsmonitorings am Robert Koch-Institut.  Die Monitoringstudien KiGGS, DEGS/GzuErn und GEDA werden in regelmäßigen Abständen durchgeführt und decken ein großes Themenspektrum ab \autocite{KurthBärbel-Mariaundweitere2009}. Neben Prävalenzschätzungen sind auch Analysen zu Zusammenhängen zwischen Gesundheitsoutcomes und zahlreichen Determinanten möglich. Zunehmend wird in den Gesundheitssurveys die Diversität der Gesellschaft besser abgebildet, wie beispielsweise durch die verstärkte Einbindung von älteren und hochaltrigen \autocite{RobertKoch-Institut2020}, oder Menschen mit Migrationshintergrund \autocite{RobertKoch-Institut2020X}.


\subsubsection{\textbf{Indikatoren in der Gesundheitsberichterstattung}}\label{H9252754}


\begin{quote}



„The wisdom, justice, and perceived legitimacy of public decisions are crucially affected by the \textbf{quality of the information }on which they are based.”

\autocite{InstituteofMedicine(U.S.)1988}


\end{quote}


Daten sprechen nicht für sich allein und müssen entsprechend transformiert und aufbereitet werden, um als Planungsgrundlage für (gesundheits-)politische Entscheidungen zu dienen. In der Gesundheitsberichterstattung greift man hierzu auf (Gesundheits-)Indikatoren zurück. Möchte man beispielsweise die Sterblichkeit infolge von Herz-Kreislauf-Erkrankungen in einer bestimmten Region auswerten, könnte sich hierfür die Anzahl der Herz-Kreislauf-Todesfälle je 100.000 Einwohner in einem bestimmten Zeitraum als Indikator eignen. 


\textbf{Indikatoren }sind somit Maßzahlen, die durch die Angabe einer oder mehrerer Bezugseinheiten gekennzeichnet sind und deren Berechnung standardisiert ist. Häufig werden Indikatoren nach verschiedenen bevölkerungsbezogenen, räumlichen oder auch zeitlichen Bezugseinheiten variiert \autocite{HamburgerProjektgruppeGesundheitsberichterstattung1998}.


\emph{Exkurs zur Erläuterung der drei Ebenen?}\emph{ }

\emph{- zeitlicher Bezug}

\emph{- Darstellungsebenen (absolut, relativ, in v.H.)}

\emph{- Gliedersungsebenen (Geschlecht, SES, Region, MIgration...)}


Indikatoren liefern dabei ein Maximum an Informationsverdichtung zu einem bestimmten Interessensgebiet. Häufig umfassen Indikatoren auch Interessenbereiche für (politische) Maßnahmen oder dienen der (politischen) Zielsetzung, zum Beispiel wenn ein kommunales Gesundheitsziel darauf abzielt, den prozentualen Anteil der jugendlichen Raucher\_innen auf weniger als 20\% zu reduzieren. Indikatoren sollten dies auf möglichst effiziente Weise tun, d.h. eine möglichst einfache Darstellung liefern (Kramers 2005 zit. nach Verschuuren et al. 2014). Traditionell wird hierzu auf numerische Darstellungsformen zurückgegriffen, in den letzten Jahren finden aber auch zunehmend moderne visuelle Aufbereitungen Anklang \emph{(--> Verweis auf Dashboards und adaptive Online-Grafiken?) }


Eine einheitliche Verwendung von Indikatoren ermöglicht Vergleiche verschiedener Bevölkerungsgruppen, Regionen und Zeitbezüge. Ziel und Zweck einer indikatorengestützten Gesundheitsberichterstattung ist die kompakte Darstellung (gesundheits)relevanter Themen um Informationsdefiziten entgegenzuwirken, Problem- sowie Optionsfelder aufzuzeigen und prioritäre gesundheitspolitische Entscheidungshilfen zu unterstützen. Im Idealfall kann die indikatorengestützte Berichterstattung somit eine evidenzinformierte Entscheidungsfindung unterstützen. 


Infobox - Zielsetzung und Verwendungszweck von Gesundheitsindikatoren (AOLG 2003:18)


\begin{tabu} to \textwidth { |X| }
\hline



- wichtiges Werkzeug zur Formulierung und Umsetzung von \textbf{Gesundheitspolitik}

\textbf{-}ermöglichen \textbf{Fortschrittskontrolle}, z.B. durch Monitoring und Evaluation der gesundheitlichen Lage

- stellen \textbf{Maßstäbe} dar und bieten so \textbf{Vergleichsmöglichkeiten} für Länder und Kommunen (Benchmarking)

- verbessern die Möglichkeit zur \textbf{Kommunikation} und \textbf{Koordinierung}

\textbf{- }können wichtige Informationen über \textbf{gesundheitspolitische Prioritäten }geben
 \\
\hline

\end{tabu}




\textbf{Indikatorensatz für die Gesundheitsberichterstattung der Länder }


Bereits vor dem Aufbau einer nationalen Gesundheitsberichterstattung wurde im Jahr 1991 die erste Version eines Länderindikatorensatzes mit 190 Indikatoren veröffentlicht. Er sollte als Grundlage für Gesundheitsrahmenberichte der Länder dienen. 1996 wurde nach ersten Erfahrungenmit dieser Datenbasis eine überarbeitete und gekürzte Version verabschiedet. Die Veröffentlichung der heute gültigen, dritten Version des Indikatorensatzes erfolgte 2003 auf Beschluss der Arbeitsgemeinschaft der Obersten Landesgesundheitsbehörden (AOLG) und im Auftrag der Gesundheitsministerkonferenzen [41]. Das heißt, die Gesundheitsministerien aller Länder haben sich auf diesen Indikatorensatz geeinigt. Die insgesamt 297 Indikatoren verteilen sich auf die Themenfelder: Bevölkerung, wirtschaftliche und soziale Lage; Morbidität und Mortalität; Gesundheitsverhalten; Gesundheitsrisiken aus der natürlichen und technischen Umwelt; Einrichtungen des Gesundheitswesens; Inanspruchnahme von Leistungen der Gesundheitsversorgung; Beschäftigte und Ausbildung sowie Ausgaben und Kosten im Gesundheitswesen. Zu jedem Indikator gibt es eine kurze Metadatenbeschreibung, die Auskunft gibt z.B. über Datenquellen, Periodizität, Aussagekraft. Bis zu 80 Indikatoren (je nach Land) liegen auch auf Kreisebene vor. Die Indikatoren werden fortlaufend aktualisiert und es gibt länderspezifische Ergänzungen. Zurzeit wird ein ergänzendes Indikatorensystem für die Präventionsberichterstattung der Länder entwickelt. [copy/paste aus Rosenkötter et al: Gesundheitsberichterstattung der Länder und Kommunen: Public Health an der Basis. Bgbl 2020 (Infobox 3)]


In der Europäischen Union sowie auf nationaler Ebene wird häufig der Europäischen Kernindikatorensatz für Gesundheit (European Core Health Indicators – ECHI) verwendet. Die ECHI-Indikatoren verfolgen einen umfassenden Public-Health-Ansatz und bilden Eckpunkte zu den Themen Demografie, sozioökonomische Lage, Gesundheitszustand, Gesundheitsdeterminanten, Versorgung und Gesundheitsförderung ab. Neben den Indikatoren mit ihrer Definition wurden ebenfalls Metainformationen wie empfohlene Datenquellen und Datentyp, Verfügbarkeit, Vergleichbarkeit erarbeitet \autocite{VerschuurenMarikeundweitere2012}


\subsubsection{\textbf{Wichtige Kennziffern der kommunalen }\textbf{Gesundheitsberichterstattung}}\label{H918331}



Gesundheitsberichte können auf verschieden Ebenen Orientierungsdaten liefern:

\begin{itemize}
\item \textbf{Absolute Fälle:} Absolute Fallzahlen sind oft für die Gesundheitsplanung wichtig. Sie geben Aufschluss über Mengengerüste, z.B. über die Größe einer Zielgruppe für Präventionsmaßnahmen, den Bedarf an Versorgungsleistungen oder vorzuhaltende Ressourcen.


\item \textbf{Prävalenz: }Die Prävalenz bezeichnet die Häufigkeit einer Erkrankung, meist in Form einer Quote. So gibt beispielsweise der Krankenstand den Anteil der zu einer bestimmten Zeit krankgeschriebenen Beschäftigten an. Bei der Prävalenz ist der Zeitbezug wichtig. Eine Punktprävalenz beschreibt die Quote der Kranken zu einem Zeitpunkt, die 30-Tage-Prävalenz die Quote derer, die im Zeitraum von 30 Tagen krank waren, die Lebenszeitprävalenz den Anteil derer, die je in ihrem Leben einmal unter der Erkrankung gelitten haben.


\item \textbf{Inzidenz: }Die Inzidenz bezeichnet die Neuerkrankungsrate. Auch hier ist der Zeitbezug wichtig. Manche gesundheitlichen Merkmale gibt es nur als Inzidenzen: Unfälle etwa oder Sterbefälle.(Sterbefall als gesundheitliches Merkmal? 


\item \textbf{Risikomaße:} Inzidenzen lassen sich auch als absolutes Erkrankungsrisiko lesen. Setzt man die Inzidenzen zweier Gruppen in Relation, spricht man vom "Relativen Risiko". Es gibt einen Hinweis darauf, ob eine Gruppe stärker von einer Erkrankung betroffen ist als eine andere. Es gibt eine Vielzahl spezieller Risikomaße, auf die hier nicht näher eingegangen werden soll, dazu sei auf die Literatur zur Epidemiologie verwiesen.


\item \textbf{Rohe und altersstandardisierte Größen:} Viele gesundheitheitliche Merkmale hängen stark vom Alter ab, z.B. Krankheitshäufigkeiten wie Bluthochdruck, Demenzen, Herzkreislauferkrankungen, oder die Sterblichkeit. Wenn man beim Vergleich zweier Gruppen wissen will, ob sie unabhängig vom Alter unterschiedlich betroffen sind, muss man den Unterschied des Altersaufbaus beider Gruppen statistisch bereinigen. Auch hier sei auf die Literatur zur Epidemiologie verwiesen. Eine allgemeinverständliche Einführung liefert die GBE-Handlungshilfe 2 "Epidemiologie und Gesundheitsberichterstattung" des Bayerischen Landesamtes für Gesundheit und Lebensmittelsicherheit (https://www.lgl.bayern.de/gesundheit/gesundheitsberichterstattung/methoden\_handlungshilfen/index.htm).  


\item \emph{\textbf{Eventuell die Handlungshilfen im Serviceanhang referenzieren und hier nur auf den Anhang verweisen? Wäre vielleicht schöner als der lange Link im Text - oder man legt den Link hinter die Bezeichnung der Broschüre?}}


\end{itemize}

\subsubsection{Ergebnisdarstellung}\label{H5142590}



\subsubsection{Formate der GBE}\label{H2213278}



Traditionell werden in der GBE Basis- und Spezialberichte unterschieden. Basisberichte haben den Anspruch, die gesundheitliche Lage der Bevölkerung umfassend darzustellen. Spezialberichte widmen sich fokussiert einem Thema: Gesundheit von Kindern und Jugendlichen, Sucht, Armut, etc. Beispiele unterschiedlicher kommunaler Gesundheitsberichte sind z. B. in der nordrhein-westfälischen Datenbank Kommunale GBE gelistet und verschlagwortet (https://www.lzg.nrw.de/gbe/).


Zu den Produkten der GBE können jedoch nicht nur Berichte gezählt werden. Zur Produktpalette der GBE  gehören auch regelmäßig gepflegte und online zur Verfügung gestellte Indikatorensysteme, Metadaten, Grafiken und interaktive Gesundheitsatlasangebote oder Dashboards (z.B. www.stuttgart.de/sozialmonitoring). 


Auch das eigentliche Berichtsformat kann nicht mehr als klassischer Printbericht verstanden werden. Auch Kurzberichte, Factsheets, online dargestellte und erläuterte Inhalte, spezielle Aufbereitungen wie policy briefs, Präsentationen für Fachausschüsse, Stellungnahmen, etc. weisen Eigenschaften der Berichterstattung auf, in dem sie faktenbasiert die gesundheitliche Lage darstellen. Auf Bundesebene wurde vom RKI eine Fachzeitschrift für Gesundheitsberichterstattung (\href{http://www.rki.de/johm}{www.rki.de/johm}) aufgebaut, in dem ein Teil der GBE-Ergebnisse veröffentlicht werden.\autocite{SaßAnkeundweitere2018}


Je nach Ressourcenverfügbarkeit und kommunalem Kontext sind weitere Formate denkbar, die die Berichterstattung ergänzen können. Hierzu zählen Blogs, Storytelling-Ansätze, Infografiken, Erklärvideos oder animierte Grafiken (GIFs) zur Vermittlung von Inhalten über Social-Media-Kanäle. Im Rahmen der englischen Anual Report Competition können unter den platzierten Berichten häufig Formate gefunden werden, die mit diesen Formaten experimentieren (https://www.adph.org.uk/our-work/about-dph-annual-report-competition/).


Zusammenfassend sollten Formate der GBE verschiedene Mechanismen bedienen  \autocite{BlessingVictoriaundweitere2017}: 

\begin{itemize}
\item push-Mechanismen: adressat\_innen-gerechte Bereitstellung von Wissen in geeigneten Formaten (zusammenfassende Instrumente, Visualisierungen (Infografiken, Karten))


\item pull-Mechanismen: z. B. interaktive online Angebote wie Datenzusammenstellungen, Gesundheitsatlanten, Analysetools, die Adressat\_innen entsprechend ihrer Bedarfe nutzen können


\item linkage/exchange-Mechanismen: regelmäßige Foren zum Austausch zwischen Berichterstatter\_innen und Adressat\_innen


\end{itemize}

\subsubsection{Datenvisualisierung}\label{H3589257}



Grafiken und Karten sind ein fester Bestandteil der GBE, um Verteilungen und Entwicklungen zu visualisieren. Eine gute Datenvisualisierung ist optisch so aufbereitet, dass relevante Muster in den Daten auffallen. Datenvisualisierungen sollten ansprechend und klar sein, nicht manipulieren und nur relevante Details darstellen \autocite{CairoAlberto2016}. Die richtige Form der Visualisierung zu finden kann genauso aufwendig sein, wie relevante Muster durch eine vertiefende Datenanalyse zu identifizieren \autocite{NussbaumerKnaflicCole2015}. Dabei sind im ersten Schritt einige einfache Grundregeln zu beachten \autocite{CairoAlberto2016} \autocite{FewStephen2012}: 

\begin{enumerate}
\item Als Grafikformate sollten Balken, Säulen, Boxen, Punkte oder Linien  gewählt werden. 


\item Da Flächen und Winkel für die Leser\_innen schwieriger zu interpretieren sind, sollten Flächen- oder Tortendiagramme nur in Ausnahmen gewählt werden. Die Länge von Balken, Säulen oder Boxen sowie die 2-D Position von Punkten sind in der Regel einfacher zu interpretieren. 


\item Der darzustellende Zahlenraum sollte in der Grafik beibehalten werden. Die Distanz der Achsenbeschriftung muss einheitlich sein und die Achsenbeschriftung sollte in der Regel bei Null beginnen - ansonsten ist die Gefahr der Manipulation groß, da Unterschiede und Trends dramatisiert oder verharmlost werden können.


\item 3-D Darstellungen sind zu vermeiden, da sie häufig nicht die Lesbarkeit der Grafik verbessern. 


\item Farben, typografische Elemente und Formen können helfen Akzente zu setzten und bestimmte Attribute in der Grafik hervorheben.


\item Farbnutzung: Farbabstufungen (ein-, maximal zweifarbig) bieten sich an, um die Verteilung intervallskalierter Variablen darzustellen. Für ordinalskalierte Variablen sollten unterschiedliche Farben genutzt werden (max. 7). Die gewählten Farben (z. B. für Geschlecht) sollten im gesamten Bericht beibehalten werden und unterschwellig vorherrschende Farbkonnationen können helfen, die Lesbarkeit der Grafik zu verbessern - zumindest sollten sie nicht vertauscht werden. Verschiedene Onlinetools helfen, Farben so zu wählen, dass sie auch bei Farbblindheit unterschieden werden können. Bei der Farbwahl ist zu bedenken, dass Rot oft als Warnung verstanden wird. Um bestimmte Darstellungen nicht zu dramatisieren, empfehlen sich eher neutrale Farbabstufungen (außer ein gewisser "aufrüttelnder" Effekt ist gewollt).


\item Vorsicht bei voreingestellten Standards der Programme, mit denen Grafiken erstellt werden. Häufig beinhalten diese Standardeinstellungen Überflüssiges oder relevante Elemente fehlen. Grafiken sollten deshalb an die jeweiligen Bedarfe angepasst und vor Aufnahme in den Bericht aufgeräumt werden. Folgende Fragen können dabei gestellt werden: Sind Gitternetzlinien notwendig oder können sie heller eingefärbt werden? Müssen die Achsen dargestellt werden oder sollte die Achsenbeschriftung angepasst werden, um die Lesbarkeit zu verbesern? Ist es sinnvoll die Datenbeschriftung (ggf. auch nur punktuell) einzufügen? Werden Farben so genutzt, dass sie die Leser\_innen an die richtige Stelle lenken und Wichtiges hervorheben? Ist es hilfreich Erläuterungen in die Grafik einzufügen (z. B. Hinweise in Zeitreihen bzgl. geänderter gesetzlicher Vorgaben)? Wäre es hilfreich die Kernaussage der Grafik in der Überschrift zu platzieren, um Leser\_innen auf relevante Muster hinzuweisen?


\end{enumerate}

Bei kartografischen Darstellungen sind weitere besondere Aspekte zu beachten, die detailliert in der \href{file:///C:/Users/OAUser/AppData/Local/Temp/ssoar-2017-augustin_et_al-Gute_Kartographische_Praxis_im_Gesundheitswesen.pdf}{Guten Kartographische Praxis im Gesundheitswesen} erläutert werden. Einige Aspekte werden an dieser Stelle exemplarisch aufgeführt: 

\begin{itemize}
\item 
\begin{itemize}
\item Die Auswahl/Festlegung der darzustellenden Raumeinheiten und die Auswahl der geeigneten Kartengrundlage. Kartengrundlagen können beispielsweise im Geoportal des  \href{https://www.geoportal.de/}{Bundesamts für Kartographie und Geodäsie} abgerufen werden. 


\item Die Auswahl des Kartentyps. Man unterscheidet je nach darzustellendem Datentyp: \textbf{Diagrammkarten} für quantitative, absolute Daten; \textbf{Choroplethenkarten} für  quantitative, relative Daten und \textbf{Standortkarten} für qualitative Daten. Darüber hinaus gibt es mehrschichtige Daten, die verschiedenen Kartentypen vereinen.


\item Farbgebung: Beim Einsatz von Farben sollte deren Assoziation berücksichtigt werden (z. B. rot für Gefahr), da hierdurch die Aussage einer Karte beeinflusst werden kann. Quantitative Daten sind durch die Variation der Helligkeit einer Farbe wiederzugeben. Bei vielen Klassen kann zusätzlich der Farbton verändert werden. Daten mit positiven und negativen Wertebereichen oder einem Schwellenwert können in einer bipolaren Farbreihe dargestellt werden. Für die Randklassen werden Komplementärfarben genutzt.


\item Klassifizierung: Es gibt verschiedene Methoden die Daten für die kartografische Darstellung zu klassifizieren. Gebräuchliche Verfahren zur Klassenzuordnung sind vor allem: konstante Breite (äquidistante) Klassen und Quantile sowie Standardabweichungen.


\item Legende und Kartenbeschriftung als erläuternde Elemente einfügen.


\end{itemize}

\item 
\begin{itemize}
\item \subsubsection{}\label{H1199864}



\end{itemize}

\end{itemize}

\subsubsection{Wie erreicht die GBE einen Impact?}\label{H6199644}



Natürlich möchte die GBE den Grundstein für faktenbasierte Entscheidungen legen. Aber nicht immer erwachsen aus Berichten umfassende Veränderungen, die dann auch noch messbar zu einer Veränderung der gesundheitlichen Lage führen. 


Aber es gibt Zwischenschritte, die ebenfalls einen Impact der GBE zum Ausdruck bringen \autocite{RosenkötterNundweitere2020}: 

\begin{itemize}
\item die Stärke und Breite der Resonanz nach Veröffentlichung von Gesundheitsberichten oder anderen GBE Produkten, 


\item die Berücksichtigung der Ergebnisse der GBE in weiteren Planungsprozessen, 


\item die Entwicklung von Strukturen oder Gremien zu einem in der Berichterstattung hervorgehobenen Sachverhalt, 


\item die Bereitstellung von Fördermitteln zur Umsetzung der abgeleiteten Handlungsempfehlungen, 


\item die Etablierung von konkreten Maßnahmen und Programmen basierend auf den Handlungsempfehlungen der GBE.


\end{itemize}

Der Impact der GBE hängt auch davon ab, ob alle wesentlichen Aspekte, die den Impact beeinflussen, bei der Planung und Umsetzung berücksichtigt werden. Dazu gehören die politische und strategische Relevanz des Berichts, die sinnvolle Integration unterschiedlicher Daten (innerhalb des Gesundheitsbereichs, aber auch ressortübergreifend), die  Zuverlässigkeit der Ergebnisse, die Qualität der Interpretation der Daten, die wissenschaftliche Basis der dargestellten Inhalte, die nachvollziehbare Darstellung von Bedarfen, die lösungsorientierte Darstellung von Handlungsfeldern und Handlungsoptionen, die kontinuierliche Interaktion mit den AdressatInnen auch während der Berichterstellung, der Zeitpunkt der Veröffentlichung im politischen Prozess, das Marketing des Berichts und die breite Nutzbarkeit der Ergebnisse \autocite{VanBon-MartensMarjaundweitere2012}\autocite{VanBon-MartensMJundweitere2019} \autocite{RosenkötterNundweitere2020}.





\subsubsection{"Vermarktung" GBE - Pressearbeit und / oder Kommunikation }\label{H5951033}






\subsubsection{\textbf{Schnittstellen zu anderen Berichtssystemen:}\textbf{ Integrierte Gesundheitsberichterstattung}}\label{H2446639}



Gesundheit und Gesundheitsrisiken hängen, neben dem Geschlecht, im erheblichen Maße vom sozialen Status, oder, weiter gefasst, von der sozialen Lage ab. Ganz allgemein haben statushöhere Bevölkerungsgruppen mehr materielle Ressourcen und soziales Kapital für eine gesunde Lebensführung, ihre Berufstätigkeit ist gesundheitlich weniger belastend und ihre Wohnsituation inklusive Wohnumfeld ist deutlich besser. Die Unterschiede zwischen der statushöchsten und der statusniedrigsten Bevölkerungsgruppe hinsichtlich Morbidität und mittlerer  Lebenserwartung sind daher beträchtlich. Dieser soziale Gradient manifestiert sich bereits in den Schuleingangsuntersuchungen (\emph{siehe oben}).


Integrierte Gesundheitsberichterstattung bietet die Möglichkeit, die Abhängigkeit der Gesundheitschancen von Faktoren, die letztendlich mit sozialer Ungleichheit zu tun haben (dazu zählt auch die Wohnlage), abzubilden und auf diese Weise die Ergebnisse eines Gesundheitsberichtes angemessen einzuordnen. Es hat wenig Sinn, die Unterschiede in der mittleren Lebenserwartung innerhalb einer Stadt darzustellen, ohne dabei auch auf die Sozialstruktur der einzelnen städtischen Quartiere und die dort herrschenden Lebensbedingungen einzugehen. 


In integrierten Berichten, die im Idealfall gemeinsam von verschiedenen Ämtern bzw. Ressorts geschrieben werden, werden neben gängigen Gesundheitsdaten auch Informationen genutzt, die die Lebenssituation der Menschen abbilden. Hierzu können eine Vielzahl an Daten zusammengetragen und erhoben werden. Diese Arbeit kann mühsam und zeitaufwendig sein, kann sich aber auch als lohnend herausstellen, da im Rahmen der sektorübergreifenden Zusammenarbeit Handlungsempfehlungen auf der Verhältnisebene abgeleitet werden können. Das niedersächsische Tool \href{https://www.kontextcheck.de/}{Kontextcheck Kommunale Gesundheitsförderung und Prävention strategisch gestalten} beinhaltet einen Leitfaden und entsprechende Handlungshilfen. 


Da viele Gesundheitsdaten wie etwa die Todesursachenstatistik keine Informationen zur sozialen Lage enthalten, ist es jedoch oftmals schwierig, im Rahmen der GBE oder integrierter Berichterstattungen sozial bedingte gesundheitliche Ungleichheit darzustellen. In solchen Fällen bieten sich raumbezogene Auswertungen an, bei der auf der räumlichen Ebene Gesundheitsdaten mit Sozialindices oder Deprivationsindices verknüpft werden. Diese Indices sind so genannte Proxyvariablen, die die Sozialstruktur der Bevölkerung räumlichen Einheiten, die der Analyse zugrundeliegen, abbilden. Bei diesem Ansatz besteht allerdings prinzipiell die Gefahr eines ökologischen Fehlschlusses. Ein ökologischer Fehlschluss liegt vor, wenn von Zusammenhängen, die auf der Kollektivebene beobachtbar sind, fälschlicherweise auf die Individualebene geschlossen wird. Um ein Beispiel zu geben: Die Feststellung, dass in Stadtteilen mit hohen Migrantenanteilen die Übergewichtsprävalenz bei einzuschulenden Kindern ebenfalls hoch ist, bedeutet nicht zwingend, dass Kinder von Migranten häufiger übergewichtig sind. Ein solcher ökologischer Fehlschluss kann etwa entstehen, wenn im zugrundeliegenden Sozialraum auch oder sogar v.a. andere Kinder häufig übergewichtig sind.   


Der \href{https://lekroll.github.io/GISD/}{German index of socio-economic deprivation} des Robert Koch-Instituts ist ein Beispiel für einen Deprivationsindex, in dem Indikatoren zur Bildung, Arbeit und Einkommen zusdammengeführt werden [ref JoHM Kroll]. Dieser Index liegt bis auf Städte- und Gemeindeebene vor. Für die kommunale Gesundheitsberichterstattung besteht die Möglichkeit, auf auf eigene Sozialindikatorensets oder selbst entwickelte Sozialindices zurückzugreifen, sofern die Kommune eine gewisse Größe aufweist. Insofern sind sozialräumliche Analysen von Gesundheitsdaten nur für größere Städte eine realistische Option. 


Um innerhalb der Kommune Mehrfachbelastungen oder Vulnerabilitäten innerhalb von Räumen darzustellen, können darüber hinaus eigene Indices gebildet werden. Vielfach werden in diesem Rahmen Daten verschiedener Ressorts wie beispielsweise Soziales, Gesundheit, Bildung, Integration, Umwelt oder Wirtschaft zusammengeführt. Dabei können schnell umfangreiche Indikatorensammlungen entstehen. Neben einer detaillierten Analyse kann es hilfreich sein zusammmenfassende Indices zu bilden. Entsprechende \href{https://www.gib.nrw.de/service/downloaddatenbank/lebenslagen-in-sozialraeumen-auf-einen-blick-indizes-in-der-kommunalen-berichterstattung}{Arbeitshilfen} können einen ersten Einblick in die Thematik liefern.








\subsubsection{Methodisch-fachliche Qualifikation}\label{H1887934}



Die methodisch-fachlichen Anforderungen an Berichterstatter\_innen sind hoch, weshalb eine Grundqualifikation im Bereich der Gesundheits-, Sozialwissenschaften oder angrenzender Fachgebiete sinnvoll ist. Gesundheitsberichterstattung bedeutet - wie dieses Kapitel aufzeigt - den Umgang mit (gesundheitsbezogenen) Daten. Eine gewisse Affinität zu Zahlen ist ebenso von Vorteil wie ein ganzheitliches Grundverständnis von Gesundheit (Verweis Stellenbeschreibung, Lit. GP GBE 2.0 2019).  Wenn Gesundheitsberichterstattung als Mehrwert aufgefasst wird und nicht als "bloßes Zusammenstellen" von Datentabellen aus verschiedenen Datenquellen, sind Kenntnisse epidemiologischer und sozialwissenschaftlicher Methoden notwendig. Das Verständnis epidemiologischer Kennzahlen, die die Verteilung von Gesundheit in der Bevölkerung und deren Determinanten aufzeigen (Rothman), ist das Handwerkszeug selbst wenn "nur" Daten aus vorhandenen Quellen für die Berichterstellung verwendet werden. Es ist die Voraussetzung, bereitgestellte Daten aus Abteilungen des Gesundheitsamtes, von eigenen Statistikstellen oder von Landes- oder Bundesbehörden zu verstehen, zu interpretieren und anderen erklären zu können. Die adressat\_innengerechte Beschreibung und (grafische) Aufbereitung der Daten erfordert die Fähigkeit komplexe Zusammenhänge zu durchdringen, um diese so darstellen zu können, dass sie für Rezipient\_innen verstehbar sind. Das Erstellen von Grafiken und Karten, die sich in der GBE aufgrund ihrer vermeintlichen Einfachheit und Klarheit großer Beliebtheit erfreuen, sollte im Vorfeld gut bedacht werden. Die Wirkung von Bildern in Form von Grafiken oder Karten darf nicht unterschätzt weden. Eine  intensive Auseinandersetzung mit der Aussagekraft der Daten und den o.g. Empfehlungen ist zwingend notwendig.


Sobald  Daten integriert werden mit Daten anderer Ressorts oder Sektoren (Verweis integrierte Berichterstattung), sind sozialepidemiologische, arbeitssoziologische, planerische oder umweltbezogene Kenntnisse notwendig, um sinnvolle Verknüpfungen zu erstellen. Hier ist es ratsam, die Berichterstattung als interaktiven Prozess im Austausch mit anderen Expert\_innen zu verstehen (Verweis Kap. Vernetzung). Interdisziplinäres, vernetztes Agieren als Arbeitsmethode und das Einlassen auf neue Themen muss für Gesundheitsberichterstatter\_innen selbstverständlich und gewollt sein.


Neben diesen methodisch-fachlichen Kompetenzen sind Kenntnisse des Verwaltungshandelns und -aufbaus (Verweis Kap. Strukturen) relevant, um beispielsweise notwendige Abstimmungsprozesse im Verlauf des Prozesses zu initiieren.


Für Quereinsteiger\_innen in den Bereich der Berichterstattung  gibt es Fortbildungsangebote der Akademie für Öffentliches Gesundheitswesen (Link Akademie Fortbildungsveranstaltungen), die von der Einführung in die Berichterstattung bis zur Vertiefung statistischer Methoden reichen. Die Inhalte der Fortbildungen orientieren sich an Bedarfen und Bedürfnissen der Berichterstatter\_innen ebenso wie an der stetigen  Weiterentwicklung der GBE. Teilweise organisieren sich die GBEler\_innen mit Unterstützung des Landesgesundheitsamtes auch selbst, um regelmäßig fachlich-kollegiale Beratung zu ermöglichen, da sie in ihren Ämtern oft als Einzelkämpfer unterwegs sind (z.B. Arbeitskreis Qualitätssicherung in der GBE in Baden-Württemberg oder die Fachtagung Kommunale GBE in Nordrhein-Westfalen). 





\subsubsection{Weiterführende Informationen:

}\label{H7188325}



\printbibliography[title={Literaturverzeichnis}]
\end{document}
