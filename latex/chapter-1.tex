\documentclass{article}

\usepackage{caption}
                
\usepackage[backend=biber,hyperref=false,citestyle=authoryear,bibstyle=authoryear]{biblatex}
                
\bibliography{bibliography}
            
\usepackage{graphicx}
                
\usepackage{calc}
                
\newlength{\imgwidth}
                
\newcommand\scaledgraphics[2]{%
                
\settowidth{\imgwidth}{\includegraphics{#1}}%
                
\setlength{\imgwidth}{\minof{\imgwidth}{#2\textwidth}}%
                
\includegraphics[width=\imgwidth,height=\textheight,keepaspectratio]{#1}%
                
}
            
\usepackage{tabu}
\begin{document}

\title{GBE01-Was ist GBE? }

\maketitle





\subsection{Was ist das für ein Buch?}\label{H2662218}



Gesundheitsberichterstattung gehört zu den Wörtern, die man oft hört und doch nicht so recht weiß, was damit gemeint ist, wie man sie gut macht und was man damit erreichen will. Das merken vor allem Einsteiger\_innen schnell, wenn sie "mal einen Gesundheitsbericht machen wollen". An wen wendet sich eigentlich so ein Bericht, was gehört inhaltlich rein, wie soll der Stil sein, woher kommen die Daten, mit wem muss ich was abstimmen ... ? Fragen, über die man am besten nicht erst dann nachdenkt, wenn man im ersten Anlauf schon einmal alles falsch gemacht hat.


Das vorliegende kleine Buch will dabei helfen, einen informierten Einstieg in die kommunale Gesundheitsberichterstattung zu finden und nicht "einfach so anzufangen". Es bleiben trotzdem genug Hürden und Stolperstellen. Es wendet sich in erster Linie an Anfänger\_innen der Gesundheitsberichterstattung, an Leute, die neu im Gesundheitsamt anfangen, an Studierende in gesundheitswissenschaftlichen Fächern. Aber auch die, die schon erste Erfahrungen mit Gesundheitsberichten hinter sich haben, finden vielleicht noch den einen oder anderen hilfreichen Hinweis.


Das Buch ist kein klassisches Lehrbuch, das auf der Grundlage von Seminarmanuskripten geschrieben wurde. Es ist als "book sprint" entstanden. Vom 15. bis zum 17. Juli 2020 haben sich die Autor\_innen zusammengesetzt und in einem moderierten Prozess Ideen gesammelt, Textbausteine fabriziert, kritisch gegengelesen, ergänzt und umgeschrieben und das an den drei Tagen mehrmals hintereinander. Technisch unterstützt wurde dieser Prozess durch das Programm "Fidus Writer", das das gleichzeitige Arbeiten an einem Dokument intelligent managt - auch Gesundheitsberichte könnte man übrigens so erstellen. Herausgekommen ist also ein Gemeinschaftswerk, dessen Passagen erstens nicht mehr klar einzelnen Autor\_innen zuzuordnen sind und das zweitens nicht "fertig" in dem Sinne ist, dass alles für alle Zeiten da steht und nicht mehr verändert werden kann. Vielmehr ist es eine 1.0-Version, die sich weiterentwickeln soll, auch durch Rückmeldungen der Nutzer\_innen: Was war hilfreich, was war unverständlich, was fehlt? Die Version 2.0 wird dann vielleicht sogar ein bisschen besser sein - und wer weiß, was uns für die Version 3.0 dann noch alles einfällt. Wie früher die Loseblattsammlungen ist das GBE-Buch ein lebendiges Werk, work in progress. Gerne laden wir alle Interessierte ein, sich kritisch und konstruktiv an diesem Prozess zu beteiligen.


Die Arbeit daran hat großen Spaß gemacht und war auch lehrreich im Kreis der Autor\_innen. Auch da sehen nicht alle alles gleich, auch da gab es Erkenntnisgewinne. Wir hoffen, dass es den Leser\_innen auch so geht, und falls nicht, dass Sie uns durch ihre Rückmeldungen helfen, die Version 2.0 so zu verbessern, dass sie dann auch Ihnen gefällt.


Berlin, 17.7.2020, das Autor\_innenteam





\subsection{Was ist Gesundheitsberichterstattung?}\label{H5307720}


\begin{quote}



„Gesundheitsberichterstattung beschreibt die gesundheitliche Lage der Bevölkerung, analysiert Problemlagen und zeigt Handlungsbedarfe für die Gesundheitsversorgung, Gesundheitsförderung und Prävention auf. Sie bietet damit eine rationale Grundlage für partizipative Prozesse und gesundheitspolitische Entscheidungen.“ (GP GBE 2019) 




\end{quote}


Gesundheitsberichterstattung (GBE) bezeichnet ein strukturiertes und datengestütztes Verfahren zur Beschreibung der gesundheitlichen Situation der Bevölkerung oder einer Bevölkerungsgruppe. Eine allgemein verbindliche Definition der Gesundheitsberichterstattung mit einer klaren Abgrenzung gegenüber anderen datengestützen Berichtsformen gibt es jedoch nicht. Was eher als Gesundheitsbericht und was eher als Aufklärungsbroschüre, amtliche Mitteilung oder als Gutachten zu klassifizieren ist, kann im Einzelfall also strittig sein. 


Gesundheitsberichte werden von unterschiedlichen Institutionen erstellt. Es gibt beispielsweise Gesundheitsberichte der Weltgesundheitsorganisation (WHO), der Europäischen Union (EU), des Robert Koch-Instituts (RKI), der Länder, der Kommunen, des Weiteren Gesundheitsberichte von Krankenkassen oder auch betriebliche Gesundheitsberichte. 


In diesem Lehrbuch stehen amtliche Gesundheitsberichte auf der kommunalen Ebene im Mittelpunkt, also Gesundheitsberichte, die von Gesundheitsämtern oder anderen kommunalen Einrichtungen erstellt werden.


kurzer hinweis kommunale/staatl. GA - danach einheitlich von GA sprechen


Rechtliche Grundlage dafür sind die Gesundheitsdienstgesetze der Länder. In Deutschland regeln die Bundesländer die Aufgaben des Öffentlichen Gesundheitsdienstes. Jedes Bundesland hat ein solches Gesundheitsdienstgesetz (in Thüringen gibt es analog eine Rechtsverordnung). Darin wird die Gesundheitsberichterstattung für nahezu alle Länder als Pflichtaufgabe des Öffentlichen Gesundheitsdienstes umschrieben, häufig mit der Zielsetzung, Maßnahmen der Gesundheitsförderung und Prävention zu unterstützen. In Bayern heißt es beispielsweise in Art. 10 des Gesundheitsdienst- und Verbraucherschutzgesetzes zur Gesundheitsberichterstattung:

\begin{quote}



\emph{Als fachliche Grundlage für die Planung und Durchführung von Maßnahmen, welche die Gesundheit fördern und Krankheiten verhüten, beobachten die Behörden für Gesundheit, Veterinärwesen, Ernährung und Verbraucherschutz aller Verwaltungsstufen sowie das Landesamt die gesundheitlichen Verhältnisse von Menschen einschließlich der Ernährung und der Auswirkungen der Umwelteinflüsse auf die Gesundheit, sammeln darüber Erkenntnisse und nichtpersonenbezogene Daten, bereiten sie auf und werten sie }\emph{aus}\emph{.}


\end{quote}


Die jeweilige Ausgestaltung sowie der vorgeschriebene Grad an Verbindlichkeit und Rahmen in dem Gesundheitsberichterstattung erfolgen soll, variieren jedoch erheblich. Dies zeigt sich insbesondere, wenn man die Gesetzestexte vergleichend gegenüberstellt (siehe Tab. 1). In manchen Bundesländern, etwa Nordrhein-Westfalen, Baden-Württemberg oder Berlin, gibt es ausführlichere Vorgaben für die Gesundheitsberichterstattung, auch Verknüpfungen zu kommunalen Gesundheitskonferenzen (KGK) oder kommunalen Gesundheitszielen. In einzelnen Bundesländern existiert hingegen lediglich eine Rahmenverordnung zur Beobachtung und Bewertung des allgemeinen Gesundheitszustandes der Bevölkerung, ohne jedwede Konkretisierung oder Spezifizierung hinsichtlich gesetzlicher Zuständigkeit, Periodizität des Berichtwesens oder inhaltlichen Vorgaben zu relevanten Datenquellen oder Indikatoren (siehe Kapitel 3). Betrachtet man zudem die gesetzlichen Vorgaben für die Landesebene und kommunale Ebene separat, zeigt sich eine noch deutlichere Heterogenität in der Ausgestaltung und den damit verbundenen Freiheitsgraden. Knapp die Hälfte aller Gesundheitsdienstgesetze beinhaltet Vorgaben zur Periodizität bzw. dem Berichtsturnus auf Landesebene, auf kommunaler Ebene (bzw. Bezirksebene) findet sich das vor allem in den Stadtstaaten. Inhaltliche Vorgaben wie Regelungen zur Datenweitergabe, aufzunehmende Themenfelder oder auch Angaben zur gewünschten Aggregationsebene sind sowohl auf Landes- wie auf kommunaler Ebene in knapp der Hälfte aller Gesetzestexte enthalten. Auf Landes- wie auf kommunaler Ebene attestiert die weit überwiegende Mehrheit aller Gesundheitsdienstgesetze der Gesundheitsberichterstattung eine gesundheitspolitischen Steuerungsfunktion (siehe Tab. 1, Quelle [\autocite{RosenkötterNicoleundweitere2020}])

\begin{figure}
\scaledgraphics{d9babbfb-61d4-4e18-abe4-76e891ab59e5.jpg}{1}
\caption*{Verankerung von Gesundheitsberichterstattung in den Gesundheitsdienstgesetzen der Länder (Stand 02/2020) | Legende: x gesetzliche Regelung, (x) gesetzliche Regelung unspezifisch, GBE Gesundeitsberichterstattung | Länderkürzel: Baden-Württemberg (BW), Bayern (BY), Berlin (BE), Brandenburg (BB), Bremen (HB), Hamburg (HH), Hessen (HE), Mecklenburg-Vorpommern (MV), Niedersachsen (NI), Nordrhein-Westfalen (NW), Rheinland-Pfalz (RP), Saarland (SL), Sachsen (SN), Sachsen-Anhalt (ST), Schleswig-Holstein (SH), Thüringen (TH)}\label{F7030531}
\end{figure}


Die Übersicht zeigt, dass Gesundheitsberichterstattung dabei nicht nur eine gesundheitsstatistische Aufgabe ist, sondern eingebettet in einen gesundheitspolitischen Handlungszusammenhang erfolgt. Im Idealfall soll gesundheitspolitisches Handeln auf einer rationalen Informationsbasis beruhen. Sie kann gesundheitspolitische Maßnahmen vorbereiten oder evaluieren, d.h. in verschiedenen Phasen eines Handlungszyklus integriert sein (siehe Kapitel 2, Public Health Action Cycle).


Zugleich soll die Gesundheitsberichterstattung gesundheitspolitisches Handeln transparent und nachvollziehbar machen. Sie erfüllt also nicht nur eine Planungs- und Evaluationsfunktion, sondern auch eine Kommunikationsfunktion. Neben dem gesetzlich vorgegebenen Rahmen hängen alle diese Funktionen jedoch auch davon ab, dass die Gesundheitsberichterstattung relevante Themen aufgreift, auf der bestmöglichen Datenbasis beruht, diese Daten auf wissenschaftlich adäquatem Niveau aufbereitet und auswertet und dann in einer allgemeinverständlichen und für die Bedarfe der verschiedenen Nutzergruppen entsprechend aufbereiteten Form präsentiert. Sie soll dies in einer sachlich neutralen Form tun, also nicht manipulativ oder überredend. Gesundheitsberichterstattung soll "sauberes Wissen" anbieten.


\begin{tabu} to \textwidth { |X| }
\hline



\textbf{Textbox: Geschichte der Gesundheitsberichterstattung}


Gesundheitsberichterstattung ist ein modernes Planungs- und Kommunikationsinstrument. Sie hilft, gesundheitliche Probleme datengestützt zu identifizieren, auf die kommunalpolitische Agenda zu setzen und gegebenenfalls auch implementierte Maßnahmen zu evaluieren. Die Gesundheitsberichterstattung ist aber auch alt: Ihre Ursprünge reichen mehr als 200 Jahre zurück. Johann Peter Frank (1745-1821), der "Stammvater" der Sozialmedizin in Deutschland, hat in seinem bekannten Werk "\emph{System einer vollständigen medicinischen Polizey" }dazu aufgerufen, medizinische Ortsbschreibungen ("medizinische Topographien") zu erstellen. Sie sollten Informationen zum Klima, zur örtlichen Geologie und Pflanzenwelt, den Wohnverhältnissen, zu Kleidung und Ernährung sowie den sonstigen Lebensbedingungen dokumentieren. Ortsbeschreibungen dieser Art entstanden Ende des 18. Jahrhunderts und Anfang des 19. Jahrhunderts in vielen europäischen Städten und auch darüber hinaus. Sie wurden in Buchform publiziert. Dieses Format verschwand im 19. Jahrhundert zunehmend wieder. Stattdessen wurden in den Kreisarztgesetzen der deutschen Länder "Jahresgesundheitsberichte" verpflichtend, deren Grundstruktur die amtliche Medizinalstatistik bis weit in die zweite Hälfte des 20. Jahrhunderts geprägt haben. In der Weimarer Zeit haben die Jahresgesundheitsberichte Aufschluss über die Tätigkeiten der Kreisärzte und später der Gesundheitsämter gegeben.


Einen gravierenden Einbruch für die Gesundheitsberichterstattung brachte der Nationalsozialismus mit sich. Die mit dem "Gesetz zur Vereinheitlichung des Gesundheitswesens" 1934 flächendeckend eingerichteten Gesundheitsämter waren Teil der nationalsozialistischen Selektions- und Mordmaschinerie. Dies schlug sich auch in der Funktion der Jahresgesundheitsberichte und ihrer Dokumentation des "gesunden Volkskörpers" nieder. Nach dem Krieg wurden vor diesem Hintergrund die Befugnisse der Gesundheitsämter stark eingeschränkt und die amtliche Medizinalstatistik verlor weitgehend ihre praktische Relevanz. Erst in den 1990er Jahren kam es zu einer Wiederbelebung der Gesundheitsberichterstattung als Planungsinstrument, zeitgleich zur politischen Renaissance der akademischen "Public Health"-Strukturen in Deutschland \autocite{KuhnJosephundweitere2012}
 \\
\hline

\end{tabu}




\subsubsection{Der GBE-Prozess: Nur ein einfacher Prozess, oder steckt da vielleicht doch Arbeit dahinter? }\label{H7691965}



Die Routine oder Arbeitsweise der Gesundheitsberichterstattung kann mit Hilfe des Modells von Verschuuren et al. beschrieben werden \autocite{VerschuurenMariekeundweitere2019}. Das Modell basiert auf der Daten-Informationen-Knowldege-Wisdom Hierarchie von Ackoff \autocite{AckoffRL1989}.

\begin{figure}
\scaledgraphics{fb2e8480-f6ad-4142-bd74-cfdcfa3a8d6e.jpg}{1}
\caption*{Abbildung 1: Gesundheitsinformationssysteme und Routineaufgaben der GBE (nach Verschuuren et al. 2019)}\label{F92928341}
\end{figure}


In diesem Modell bildet der gewählte konzeptionelle Ansatz oder Rahmen die Basis für die Inhalte der GBE (siehe Abbildung 1).  Darauf folgt die regelmäßige Sammlung von Daten, die routinemäßige Aufbereitung der Daten (Informationen), die Berichterstattung (Knowledge) sowie Maßnahmen das generierte Wissen zu Verbreiten und somit die Grundlage für evidenzbasierte (oder evidenzinformierte) Entscheidungen zu liefern.


Was steckt genau hinter den einzelnen Elementen der Informationspyramide und welche Aktivitäten sind damit verknüpft?


\textbf{Konzeptioneller Rahmen}


Der konzeptionelle Rahmen steckt die Inhalte der GBE ab, was soll im Rahmen der GBE dargestellt werden und warum? Geeignete Beispiele für konzeptionelle Rahmenmodelle sind das Modell von Lalonde \autocite{LalondeMarc1974} oder das Modell der Gesundheitsdeterminanten (--> siehe Kapitel Wozu GBE?). Die Festlegung auf einen konzeptionellen Rahmen hilft dabei die GBE nicht nur von der Datenverfügbarkeit zu denken, sondern die relevanten Themen für die (gesundheits)politische Gestaltung in der Kommune im Blick zu behalten und ggf. alternative Datenquellen zu generieren um diese Inhalte abbilden zu können. 

\begin{figure}
\scaledgraphics{dddd6380-8cb3-4fc2-957b-6e99978a32b7.jpg}{1}
\caption*{Abbildung 2: Lalonde Modell "Health Field Concept" (1974)}\label{F76014001}
\end{figure}


\textbf{Daten}


Die allgemeine Datenbasis der GBE besteht aus repräsentativen Routinestatistiken. Viele der genutzten Daten stammen aus amtlichen Statistiken (z. B.  Bevölkerungsstatistik, die Todesursachenstatistik, die Krankenhausdiagnosestatistik, Mikrozensus etc.). Diese werden wurden i.d.R. nicht für den Zweck der GBE generiert, sie haben deshalb ihre eigenen Charakteristika, Anwedungsgebiete, Limitationen und Fehlerquellen, die den Gesundheitsberichterstatter\_innen sowie den Rezipient\_innen bekannt sein müssen, um Fehlinterpretationen zu vermeiden. Darüber hinaus können auch amteigene Daten, wie beispielsweise Ergebnisse der Schuleingangsuntersuchungen, die von den Gesundheitsämtern direkt erhoben werden, oder Surveydaten genutzt werden. 


Die Daten werden häufig anhand festgelegter Standards in Indikatoren (z. B. Mittlere Lebenserwartung, Säuglingssterblichkeit) überführt. Der Indikatorensatz der Ländergesundheitsberichterstattung, der auch Indikatoren auf Ebene der Kreise und kreisfreien Städte enthält, bildet in Deutschland die Grundlage für die GBE. 


\textbf{Informationen}


Damit aus den Daten und Indikatoren relevante Informationen werden, müssen sie in einen Kontext gesetzt werden: Zeitreihen, regionale Vergleiche, Vergleiche zwischen Bevölkerungsgruppen. Aus den Vergleichen, die im Rahmen des Monitorings routinehaft erfolgen, entstehen Informationen zu gesundheitlichen Unterschieden oder Vulnerabilitäten, die im Rahmen der Berichterstattung weiter aufbereitet werden können. Viele Landesämter und auch einige Kommunen stellen hierfür auch interaktive Tools wie Gesundheitsatlanten auf ihren Internetseiten bereit.


Das Setzen in einen Kontext kann nicht nur auf inhaltlicher Ebene geschehen. Neben der inhaltlichen Dimension ist es wichtig Prozesse im Blick zu behalten. Gab es beispielsweise Veränderungen die die Generierung der Statistik beeinflussen (z.B. gesetzliche Veränderungen, Änderung der Codierung der Todeursachenstatistik in den Landesstatistikämtern \autocite{EckertOlafundweitere2018}, Einführung des Erinnerungsverfahrens zur Teilnahme an den U-Untersuchungen).


\textbf{Wissen}


In der nächsten Stufe der Informationspyramide geht es um die Generierung von Wissen. In diesem Schritt soll die Nutzbarkeit für die Adressat\_innen weiter  verbessert und zusätzliche wissenschaftliche Evidenz ergänzt werden. Leitende Fragen sind (1) warum sind gesundheitliche Unterschiede zu beobachten und (2) was kann getan werden, um die gesundheitliche Chancengleichheit zu verbessern. In dieser Phase entsteht der Gesundheitsbericht, in dem die Informationen und das Wissen präsentiert und kommuniziert werden (siehe auch Formate der GBE). 


\textbf{Evidenzbasierte (-informierte) Entscheidungen}


Damit die Arbeit der Gesundheitsberichterstattung einen Impact erzeugt ist jedoch mehr nötig als die Erstellung eines Berichts. Das Wissen muss über unterschiedliche Kanäle transportiert und kommuniziert werden. Hierzu gibt es unterschiedliche Ansätze (evidence briefs, interaktive Ansätze (z.B. serious gaming), Netzwerkaktivitäten), die unter dem Schlagwort "Knowledge-translation" oder Wissenstransfer verortet werden.


 







\begin{quote}






\end{quote}


\printbibliography[title={Literaturverzeichnis}]
\end{document}
