\documentclass{article}

\usepackage{hyperref}
                
\usepackage[backend=biber,hyperref=false,citestyle=authoryear,bibstyle=authoryear]{biblatex}
                
\bibliography{bibliography}
            
\begin{document}

\title{GBE08-Wo will die  GBE hin (+Ideenspeicher) }

\maketitle





Gesundheitsberichterstattung hat das Ziel, eine verlässliche Informationsgrundlage für Entscheidungsprozesse und Maßnahmen der Akteur\_innen in Gesundheitspolitik und Public Health zu schaffen, um so einen Beitrag zur Verbesserung der Gesundheit der Bevölkerung und einzelner Bevölkerungsgruppen zu leisten. Wie wichtig verlässliche Informationen für gesundheitspolitische Maßnahmen  auf Bundes-, Länder- und kommunaler Ebene sind, wurde nicht zuletzt bei der aktuellen Corona-Pandemie und ihren Folgen deutlich. "Sauberes Wissen" ist eine wichtige Ressource, wenn es darum geht, Präventionsmaßnahmen gemeinsam mit der Bevölkerung umzusetzen. Dazu ist Vertrauen in die Informationen, die Behörden liefern, unverzichtbar. Gute Daten und ihre wissenschaftlich fundierte Aufarbeitung sind eine Voraussetzung dazu.  So gesehen, kann die Gesundheitsberichterstattung nicht nur Daten für Gesundheitspolitik bereitstellen, sondern auch die Gesundheitskompetenz der Bürger\_innen unterstützen.


In den vorangegangenen Kapiteln wurde die Arbeitsweise und das Handwerkszeug für die GBE dargestellt. Dabei wurde vielfach eine Idealvorstellung skizziert, die z. T. aufgrund fehlender Ressourcen, nicht flächendeckend sichergestellt werden kann.  Daraus ergeben sich sehr konkrete Zielvorstellungen aber auch weitergefasste Ziele die gemeinsam diskutiert werden sollten.


\subsubsection{Konkrete Zielvorstellungen}\label{H667089}



...abzuleiten aus den vorherigen Kapiteln


\subsubsection{Weitergefasste Ziele }\label{H980925}



Die Datengrundlagen der GBE entwickeln sich laufend weiter. Aktuell lässt sich beispielsweise eine starke Dynamik bei den Nutzungsmöglichkeiten der GKV-Daten beobachten, wodurch sich viele Optionen für kleinräumige Analysen und Zeitreihen ergeben. Auch die Entwicklungen bei Big Data und die Nutzung von Künstlicher Intelligenz werden sicherlich in der Gesundheitsberichterstattung eine wichtige Rolle spielen. Die Nutzung dieser Datenquellen erfordert dabei zusätzliche GBE-Ressourcen, sowohl auf der technischen IT-Ebene als auch im Bereich Personalressourcen. 


Auch die Kommunikationsformen entwickeln sich stetig weiter: Der Ausbau von interaktiven Daten- und Visualisierungsmöglichkeiten stellt eine wichtige Chance efür die GBE dar. Durch Entwicklung gemeinsamer Standards, Indikatoren und Empfehlungen würden sich zusätzliche Synergien der Gesundheitsberichterstattung auf der kommunalen, Länder und Bundes-GBE ergeben. Durch Nutzung gemeinsamer Gesundheitsinformationssysteme wären relevante Gesundheitsdaten in einheitlicher Form verfügbar und könnten von unterschiedlichen Nutzer\_innen nach gemeinsamen Standards genutzt werden. 


Auch für die gemeinsame Erstellung von Berichten bieten sich unterschiedliche Plattformen an, um gemeinsam Projekte zu planen, zu analysieren und zu publizieren (wie beispielsweise book sprint). Hier bestehen auch Schnittstellen zu möglichen anderen Berichtsformaten, z.B. Blogs, die den Leser\_innen neue Möglichkeiten zur Beteiligung an der gesellschaftlichen Debatte über die Frage, "wie geht es den Menschen", einräumt. Die neuen medialen Instrumente könnten so auch einen Beitrag zur Realisierung der Idee einer "partizipativen Gesundheitsberichterstattung" (\textbf{Literaturref. Jordan RKI, Partkom}) leisten.


Die GBE will auch die sozialen Medien und verschiedene Plattformen stärker für die Ergebnisdisseminierung nutzen. Zum einen werden so zusätzliche Nutzer\_innen der GBE erreicht, zum andern könnte so die GBE mit ihren verlässlichen Daten einen Kontrapunkt zu fake news im Internet darstellen. 


Die Weiterentwicklung der  Gesundheitsberichterstattung setzt eine enge interdisziplinäre Zusammenarbeit voraus: So müssen Erkenntnisse und Anforderungen der Public-Health- sowie der Kommunikationswissenschaften,  sowie der Data Sciences integriert werden. Auch ist es wichtig, das Thema der Gesundheitsberichterstattung in Forschung und Lehre der Gesundheitswissenschaften weiter zu platzieren. Ein wichtiger Baustein ist dabei der Ausbau der Zusammenarbeit zwischen Universitäten, Hochschulen und dem Öffentlichen Gesundheitsdienst , wie sie auch in einer aktuellen Ausschreibung des BMG\autocite{BMBF}  gefördert wird 


Neben der Weiterentwicklung der Gesundheitsberichterstattung auf der wissenschaftlichen Ebene stellt auch der Ausbau von Partizipation bei der Themenwahl und der Berichterstellung eine künftige Aufgabe dar.


Ein weiteres wichtiges Potenzial der Gesundheitsberichterstattung liegt in der Stärkung der internationalen Betrachtung von Gesundheitsdaten und Methoden für die GBE. In fast allen Ländern der EU besteht ein nationales GBE-System. Für grenzüberschreitende  Analysen und Berichte wäre es wichtig, im EU-Raum eine Einrichtung für Public Health zu schaffen, in der internationale GBE angesiedelt werden kann. Auch auf der globalen Ebene sollte die Zusammenarbeit mit der WHO gestärkt werden. Aus dieser Zusammenarbeit ergeben sich häufig Ideen, die auch für die GBE und Planung in Deutschland hilfreich sein können. Dies hat den Vorteil, dass das Rad nicht ständig neu erfunden werden muss. Ein konkretes Beispiel ist das schottische Place Standard Tool, dass Kommunen dabei unterstützt Einschätzungen des alltäglichen Lebensumfelds der Bürger\_innen in einem standardisierten Verfahren zu sammeln und prioritäre Handlungsfelder für verhältnispräventive Maßnahmen zu identifizieren. Das Tool ist bereits in vielen europäischen Ländern auf Anerkennung gestoßen [\href{http://eurohealthnet-magazine.eu/talking-place-a-public-health-conversation-for-everyone/}{ref}]. In Deutschland wurde das Tool von der Bundeszentrale für gesundheitliche Aufklärung in Kooperation mit den Ländern Baden-Württemberg und Nordrhein-Westfalen und vier Pilotkommunen übersetzt und erprobt und steht als StadtRaumMonitor voraussichtlich ab 2021 bundesweit zur Verfügung.


Die Gesundheitsministerkonferenz hat 2018 ein modernes Leitbild für den ÖGD verabschiedet (https://www.gmkonline.de/). Es soll dabei helfen, dem ÖGD ein zukunftsorientiertes Profil zu geben und dabei neben den klassischen Aufgaben der Überwachung und Kontrolle, etwa im Infektionsschutz oder der Krankenhaushygiene, auch den Aufgaben der Koordination, Vernetzung und Planung mehr Stellenwert zuzumessen.  Im Leitbild des Öffentlichen Gesundheitsdienstes\autocite{BVÖGD2018} gehört der Bereich „Kommunikation, Moderation, Anwaltschaft und Politikberatung“ zu den Kernaufgaben des ÖGD. Die Gesundheitsberichterstattung ist dabei ein wichtiges Instrument dieser Kernaufgaben. Mit ihren unterschiedlichen und neuen  Kommunikationskanälen kann die GBE dabei noch besser beitragen, die Evidenzbasierung der (gesundheits-)politischen Willensbildung und Entscheidungen zu stärken.


Die Weiterentwicklung der Gesundheitsberichterstattung mit der Nutzung neue Möglichkeiten und Antwort auf neue Anforderungen sind Zukunftsaufgaben, die nicht mit den bestehenden Kapazitäten der Gesundheitsberichterstattung realisiert werden könnten. Um auch weiterhin verlässliche Informationen zur Gesundheit für unterschiedliche Nutzer\_innen bereit stellen zu können, ist ein Ausbau der Gesundheitsberichterstattung auf allen Ebenen (Bund, Länder, Kommunen) erforderlich. Ein starkes Public-Health-System und Gesundheitsberichterstattung als dessen fester Bestandteil sind erforderlich, um die Gesundheit aller Menschen in Deutschland schützen und zu verbessern. 





Empfehlungen


Als Idee ist hier überlegt, die Kernaussagen oder Empfehlungen aus den Kapiteln hier zu zusammenzustellen. Das kann jedoch erst im Rahmen der Schlussredaktion erfolgen.


\printbibliography[title={Literaturverzeichnis}]
\end{document}
