\documentclass{article}

\usepackage{hyperref}
\usepackage{caption}
                
\usepackage[backend=biber,hyperref=false,citestyle=authoryear,bibstyle=authoryear]{biblatex}
                
\bibliography{bibliography}
            
\usepackage{graphicx}
                
\usepackage{calc}
                
\newlength{\imgwidth}
                
\newcommand\scaledgraphics[2]{%
                
\settowidth{\imgwidth}{\includegraphics{#1}}%
                
\setlength{\imgwidth}{\minof{\imgwidth}{#2\textwidth}}%
                
\includegraphics[width=\imgwidth,height=\textheight,keepaspectratio]{#1}%
                
}
            
\begin{document}

\title{GBE07-Planung}

\maketitle


Die GBE ist kein Selbstzweck, sondern sie verfolgt das Ziel, so gesicherte und "saubere" Informationen wie möglich für die Planung und Umsetzung sowie idealerweise das Monitoring gesundheitsschützender wie gesundheitsförderlicher Maßnahmen bereit zu stellen. GBE grenzt sich von der Medizinalstatistik gerade auch durch ihren Handlungsbezug ab, d.h. dem Selbstverständnis nach sollten Daten für Taten generiert und interpretiert werden (siehe Kap. 1...). Ihr Anspruch der Planungsrelevanz macht es erforderlich, die bestehenden Planungskontexte, -ziele, -strukturen und -ebenen im Blick zu haben. Im Public Health Action Cycle wird die Rolle der GBE im Planungsprozess visualisiert (siehe Abb. in Kap. 2...). Als GBE-ler\_in gilt es daher, den Planungsprozess zu kennen und immer 'mitzudenken', ähnlich wie Akteur\_innen aus der Planung (sowohl Gesundheitsplanung als auch andere kommunale Planungen) die GBE als Assessment- und Evaluationsinstrument mitdenken sollte.


In diesem Kapitel widmet sich der erste Teil der Gesundheitsplanung im engeren Sinne, d.h. der federführend durch den ÖGD durchgeführten Planung, wie sie in einigen Bundesländern auch gesetzlich verankert ist. Im zweiten Teil nimmt  dieses Kapitel auch Planungsprozesse in den Blick, die anderen Ressorts obliegen und durch die GBE unterstützt oder erweitert werden können. (siehe HiaP ...).

\begin{figure}
\scaledgraphics{a8184739-380b-436d-8469-67dcbd9664d5.png}{1}
\label{F5832671}
\end{figure}


\subsection{Gesundheitsplanung}\label{H1599355}



In einer zunehmenden Zahl an Bundesländern sind dem ÖGD mittlerweile Planungsaufgaben landesgesetzlich zugewiesen (siehe GDG ...), in weiteren erfolgen diesbezügliche Modellprojekte. Die Planung wird häufig in spezifischen Planungsgremien institutionalisiert, die vom ÖGD einzurichten und zu unterhalten sind, z.B. Kommunalen Gesundheitskonferenzen (KGK). Diese KGK sind meist als kommunale Expert\_innengremien unter Geschäftsführung durch den ÖGD angelegt und sollen planerisch handeln, häufig mit einem direkten Bezug zum Public Health Action Cycle\autocite{Hollederer2014}. Beispielhaft dafür steht u.g. Definition der Gesundheitsplanung in Baden-Württemberg:

\begin{quote}



"Die Gesundheitsplanung ist ein langfristig angelegter interdisziplinärer Planungsprozess im Rahmen der Kommunalen Gesundheitskonferenz auf Ebene von Land- und Stadtkreisen und deren Städte, Gemeinden, Stadt- und Ortsteile entlang des Public-Health-Action-Cycles zu den Handlungsfeldern Gesundheitsförderung und Prävention, medizinische Versorgung sowie stationäre und ambulante Pflege. Die Gesundheitsplanung beinhaltet die datengestützte und bedarfsgerechte Festlegung von Handlungsempfehlungen, Zielen und Maßnahmen sowie deren Umsetzung und Evaluation in den genannten Handlungsfeldern." \autocite{Albrichundweitere2017}


Expert\_innengremien wie die KGK sind keine demokratisch legitimierten Institutionen und werden daher innerhalb der kommunalen Strukturen parallel zu den kommunalpolitischen Ausschüssen geführt (siehe Strukturen ...\autocite{Hollederer2014}). Ihre Funktion ist daher normalerweise beratend, auch wenn nicht wenige nach einer gewissen Etablierung über ein vergleichsweise begrenztes Budget verfügen, das eigenständig verplant werden kann. Aufgrund des Fehlens strenger Vorgaben variiert die Zusammensetzung der KGKen etwas, aber die Expert\_innen entstammen üblicherweise den Feldern Gesundheitsversorgung, Prävention und Gesundheitsförderung. Entsprechend umfangreich ist auch die thematische Breite, denen sich die Planungsgremien widmen.


\end{quote}


Für den Aufbau einer Gesundheitsplanung ist es wichtig, möglichst gute Kenntnisse über die strukturellen Bedingungen vor Ort zu gewinnen (siehe Strukturen ...) und sich aktiv zu vernetzen (siehe Netzwerkarbeit ...). Eine GBE-Basis kann einen wichtigen Ausgangspunkt für den Aufbau einer Gesundheitsplanung darstellen, andernfalls wird fast immer im Laufe der Zeit aus diesen Strukturen der Ruf nach Daten für Taten laut, d.h. die Planung stellt die Ausgangsbasis für die GBE. Um die Planung nachhaltig erfolgreich zu gestalten, gilt es, Gestaltungsspielräume zu nutzen, die sich selbst in Bundesländern, in denen planerische ÖGD-Aufgaben in den Gesundheitsdienstgesetzen definiert sind, ergeben. Auf einige für das Gelingen wesentliche Faktoren wird im Folgenden eingegangen:  


\emph{\textbf{Mandatierung: }}Gesundheitskonferenzen sind als beratende Gremien angelegt. Um auch größere Ressourcenentscheidungen beeinflussen zu können, sollten Schnittstellen zu den kommunalpolitischen Gremien aufgebaut werden (siehe Strukturen ...). In der KGK selbst gilt es ihr Mandat frühzeitig und klar zu kommunizieren, um späterer Frustration bei denjenigen vorzubeugen, denen die Rolle der KGK im kommunalpolitischen Geschehen nicht schon per Amt bekannt ist. 


\emph{\textbf{Planungsgegenstand: }}Die gesetzlichen Vorgaben bzgl. der Planungsthemen sind üblicherweise vergleichsweise vage gehalten und eher als breite Korridore mit den übergeordneten Schwerpunkten Gesundheitsförderung, Prävention und Gesundheitsversorgung angelegt. In einigen Bundesländern sind auf Landesebene Gesundheitsziele verabschiedet worden, die als konkrete Schwerpunktthemen dienen können. Diese Offenheit ist einerseits sinnvoll, damit es möglich ist, Schwerpunkte z.B. auf Basis der im Rahmen der GBE ermittelten lokalen Bedarfslage setzen zu können. Der Gesundheitsplanung fehlt dadurch andererseits jedoch eine klare Zuständigkeit für ein Themenfeld, was jede Themenwahl potentiell beliebig und Planungen dadurch weniger durchsetzungsstark macht. Für die Themenwahl ist es daher günstig, diese einerseits datenbasiert, d.h. auf GBE-Basis, zu legitimieren. Sinnvoll ist es darüber hinaus, die lokalen Strukturen zu berücksichtigen,  d.h. konkret die Ausrichtung des zuständigen kommunalpolitischen Ausschusses im ÖGD-Dezernat sowie auch die Zusammensetzung des lokalen Expert\_innengremiums wie beispielsweise der kommunalen Gesundheitskonferenz, sofern dieses bereits etabliert ist (siehe Strukturen ...). Bestehende kommunalpolitische gesundheitsbezogene Schwerpunktsetzungen können einen wesentlichen Ausgangspunkt darstellen (z.B. lokale Gesundheitsziele oder Leitbild Gesunde Kommune, ggf. Mitgliedschaft im Gesunde Städte-Netzwerk Deutschland). Im Sinne von HiAP ist es sinnvoll, auch bei der Themenfindung stets Integrationspotentiale der Kommunalverwaltung zu nutzen und zu befördern, etwa durch Einbindung von Sozial-, Bildungs-, Behindertenhilfe-, Altenhilfe-, Jugendhilfe-, Wohnungsbau- und Stadtentwicklungsplanung\autocite{LutheE.-W.2010}. Konkret findet die Themenfindung sinnvollerweise in einem Aushandlungsprozess zwischen kommunalen Expert\_innen, Bürger\_innen und Patient\_innen sowie Informationen aus der GBE statt (siehe Abb. …, nach \autocite{RollerG.undweitere2010}).


\begin{center}
\begin{figure}
\scaledgraphics{d775a334-b272-49b5-b9b3-1338b0755261.png}{0.75}
\caption*{Kommunale Themenfindung im Trialog zwischen GBE, Expert\_innen und Laienbeteiligung (erweiterte Darstellung nach: MfAS-BW (2010) AG Standortfaktor Gesundheit des Gesundheitsforums BW, RT)}\label{F57296601}
\end{figure}


\end{center}


\emph{\textbf{Konkrete Zusammensetzung der Planungsgremien: }}Auch beim Aufbau der kommunalen Gesundheitskonferenz oder vergleichbarer gesundheitsbezogener, idealerweise intersektoral ansetzender Planungs- und Steuerungsgremien ergeben sich meist nicht unerhebliche Spielräume. Sinnvoll ist es für ihre Zusammensetzung die gewünschten zukünftigen Themenschwerpunkte zu antizipieren. Dies sollte mindestens insoweit geschehen, dass das steuernde Gremium eine tendenziell unterschiedliche Struktur aufweisen sollte in Abhängigkeit davon, ob eher Themen der Gesundheitsförderung oder Fragen zur Gesundheitsversorgung behandelt werden sollen. Da das diesbezügliche Dilemma nicht vollständig aufzulösen ist, können die Gremien nochmals in Unterarbeitsgruppen aufgeteilt werden\autocite{WollenbergBundweitere2019}. Die Aktivitäten fast aller Mitglieder in einer KGK erfolgen freiwillig. Ein weiteres Kriterium kann und sollte daher das institutionelle bzw. persönliche Interesse an einer aktiven Mitwirkung im Planungsgremium darstellen, um die konstruktive Dynamik in einer KGK auch nachhaltig gewährleisten zu können. 


\textbf{Form des Berichtswesens: }GBE ist zwar in allen Gesundheitsdienstgesetzen verankert, sie wird aber nicht überall durchgeführt, und die Aufgabe ist eher selten durch klare Vorgaben präzisiert. Nicht selten führt die Etablierung einer kommunalen Gesundheitsplanung im zweiten Schritt zur Intensivierung der GBE, da die o.g. vage thematische Zuständigkeit der Planungsgremien dadurch erheblich an rationaler Legitimation gewinnt. Sinnvoll ist häufig ein Zusammenspiel von sowohl fachlichen als auch politischen Gremien und GBE, indem Berichtsimpulse aus den Gremien aufgenommen und Berichte wiederum z.B. an die Gesundheitskonferenz oder Landkreistag adressiert werden. Abhängig von der Zusammensetzung der Expert\_innengremien ist es dennoch sinnvoll, sich in der GBE Freiheitsgrade zu erhalten, um auch anderweitige Impulse geben zu können.


\textbf{Ressourcensteuerung}: Die zunächst mal meist vagen Zuständigkeiten der durch den ÖGD eingerichteten Planungsgremien spiegeln sich auch auf der Ressourcenebene wider\autocite{Hollederer2014}. Planung bedeutet themenbezogene Ressourcensteuerung, bei diesen Ressourcen handelt es sich meist um Finanzmittel oder aber Arbeitsschwerpunkte von Mitarbeiter\_innen. Gesetzlich ist eine Zuständigkeit für eine solche Ressourcensteuerung nicht angelegt, d.h. die KGK muss sich diese Rolle und somit streng genommen ihre Relevanz meist erst erarbeiten. Grundsätzlich bestehen hierfür verschiedene Möglichkeiten. Häufig wird die KGK mit einem allerdings vergleichsweise begrenzten eigenen Budget aufgebaut, mit dem Schwerpunkte verfolgt werden können. Um nachhaltig eine befriedigende Arbeit des Gremiums zu gewährleisten, ist dies jedoch üblicherweise zu knapp bemessen. Daher werden häufig parallel kommunale Planungen beratend angestoßen und unterstützt, über die dann zuständigkeitshalber in den kommunalpolitischen Gremien entschieden wird (siehe Strukturen …). Nicht selten werden darüber hinaus themenbezogen Ressourcen einzelner Mitglieder der KGK akquiriert, häufig gelingt es auch, Mischfinanzierungen aus mehreren der o.g. Quellen anzustoßen. Eine wertvolle Möglichkeit hat sich diesbezüglich durch das Präventionsgesetz (PrävG) ergeben, indem gesetzliche Krankenversicherungen einen Teil ihrer PrävG-Mittel in Präventionsprojekte der KGK einspeisen können. Dies ist jedoch bisher nur in wenigen Bundesländern institutionalisiert und hängt somit bisher stark von der Kooperation der Akteure vor Ort ab. Für die durch den ÖGD betriebenen Planungsgremien wäre es sehr wertvoll, wenn dies bundesweit institutionalisiert werden könnte, da damit eine eindeutige Zuständigkeit der Planungsgremien für eine umgrenzte Ressourcensteuerung fixiert würde. Inhaltlich wäre dies sehr sinnvoll, da der kommunale ÖGD einen Zugang zu wesentlichen Lebenswelten ermöglicht. Es würde aber auch einen nachhaltigen Betrieb der KGK deutlich vereinfachen und die Etablierung bundesweiter Standards für die Planung ermöglichen\autocite{Szagunundweitere2016},\autocite{Starkeundweitere2018}.


\subsubsection{Nachhaltige Relevanz der Gesundheitsplanung durch den ÖGD}\label{H2024211}



Im Fazit zum Stand der Gesundheitsplanung mittels Gesundheitskonferenzen von 2015 heißt es:

\begin{quote}



"Die Gesundheitskonferenzen bieten viele Ansatzpunkte für ein kommunales oder regionales Gesundheitsmanagement, bergen aber das Risiko geringer Wirksamkeit. Sie können mangels Regulierungskompetenzen die bestehenden Interessens- und Verteilungskonflikte und Systemprobleme nicht aufheben, sondern diese nur konsensual in eigener Zuständigkeit über die Einflussmöglichkeiten ihrer Mitglieder abmildern. Die Gesundheitskonferenzen ergänzen lediglich die vorhandenen Steuerungssysteme und Strukturen. Sie benötigen als „harte“ Einflussfaktoren ausreichende Anschubinvestitionen, Ressourcen und günstige Rahmenbedingungen. Der Erfolg ist darüber hinaus stark von „weichen“ Faktoren wie Vorsitz und Führung, Moderation, Geschäftsstelle sowie vom Agenda-Setting abhängig."\autocite{Hollederer2014}


\end{quote}


Entsprechend gilt es für den nachhaltigen Betrieb einer Gesundheitskonferenz, v.a. ihre Wirksamkeit und damit Relevanz sicherzustellen, um Frustration und Ermüdungserscheinungen bei den beteiligten Akteuren zu vermeiden. Aufgrund der Rahmenbedingungen v.a. bzgl. Zuständigkeit und Finanzierung ist dies herausfordernd, aber der Aufwand lohnt allein schon aufgrund der verbesserten organisations- und professionsübergreifenden Kooperation\autocite{Hollederer2016}. Wesentliche Einflussgrößen darauf, die beim Betrieb des Gremiums beachtet werden können, liegen im politischen Gewicht, der wahrgenommenen Professionalität und dem Einbezug der Öffentlichkeit (siehe Abb. ...).

\begin{figure}
\scaledgraphics{ae8130d5-5ce9-46a2-ab35-d904dae29565.png}{1}
\caption*{Einflussgrößen auf die Relevanz einer kommunalen Gesundheitskonferenz als Planungsgremium mit begrenzter Zuständigkeit}\label{F78368531}
\end{figure}

\begin{itemize}
\item Politisches Gewicht: Das politische Gewicht lässt sich durch die Leitung und sichtbare Zuständigkeiten bzw. Ressourcenverantwortung des Gremiums erhöhen. Bestenfalls erfolgt die Leitung der KGK durch die kommunale Spitze, wodurch gleichzeitig die Schnittstelle zu den kommunalpolitischen Gremien optimiert wird. Auch der personelle Ressourceneinsatz zum Betrieb der KGK sowie eine eigenverantwortliche Ressourcenverantwortung erhöhen sichtbar nach außen das Gewicht, dass ihr seitens der Kommunalpolitik beigemessen wird\autocite{Albrichundweitere2017}.


\item Professionalität: Eng mit o.g. Ressourceneinsatz verbunden ist die Professionalität der Arbeit des Planungsgremiums und seiner Grundlagen. Dazu gehört einerseits der Einsatz von Personal, das hinsichtlich Qualifikationsgrad und Professionalität als Gegenüber zu den vielfach hochqualifizierten Mitgliedern von KGKen geeignet ist. Andererseits dient auch die GBE als Investition in die Professionalität der Arbeit und Rationalität der Diskussionen innerhalb des Gremiums. Weitere Professionalisierungsoptionen ergeben sich durch die sichtbare Integration anderer kommunaler Planungsfelder in die KGK und die aktive Mitwirkung der Amtsleitung an den Planungsprozessen.


\item Einbezug der Öffentlichkeit: Da institutionelle Interessen- und Zuständigkeitskonflikte innerhalb der Planungsgremien nicht aufgehoben sind, spielt auch der Einbezug der Öffentlichkeit eine Rolle für ihren nachhaltigen Betrieb. Eine destruktiv wirkende Dominanz von Partikularinteressen kann in von der Öffentlichkeit abgeschotteten Expertengremien ungeschminkter zutage treten. Das gezielte Hinzuziehen von Öffentlichkeit kann es somit erleichtern, Gemeinwohlinteressen gegenüber Partikularinteressen in der Gremienarbeit mehr Gewicht zu verleihen. Diese Öffentlichkeit kann durch mediale Präsenz sowie den Einbezug von Bürger\_innen und Patient\_innen bis auf die Arbeitsebene hergestellt werden\autocite{Szagunundweitere2002}.


\end{itemize}

\subsubsection{Exkurs: lokale Fachplanung Gesundheit}\label{H4813093}



Bei der hier beschriebenen Gesundheitsplanung handelt es sich um ein prozessorientiertes Instrumentarium, welches aus dem ÖGD heraus entwickelt wird und bei dem der ÖGD als plandender Initiator in Erscheinung tritt. Um diese Planung zu systematisieren, wurden in der Vergangenheit in einigen Bundesländern verschiedene Instrumente entwickelt. Als ein Beispiel wird nachfolgend das Konzept zum Fachplan Gesundheit vorgestellt, welches seit 2009 in Nordrhein-Westfalen entwickelt und erprobt wird. 


M.E. sollten wir die Fachplanung als Modellprojekt kenntlich machen, d.h. in Exkurs-Kasten o.ä. (kein Lernstoff). Besser noch wäre es, das Kapitel ‘wo-will-die-gbe-hin’ durch Modellprojekt-Part ergänzen (als Lösung schlüssiger und nachhaltiger, da dort dann alles Mögliche hinzuwachsen könnte).


Daher mein Vorschlag: Unter ‘wo-will-die--gbe-hin’ einen solchen Abschnitt aufbauen und niedrigschwellig füllen – als Bedingung sollte eine bestimmte Zeichenzahl pro Projekt nicht überschritten werden.


Der Fachplan Gesundheit ist methodisch stark an andere, seit langem zum Teil auch gesetzlich verankerten Fachplanungen (z.B. Lärmaktionsplanung, Landschaftsplanung, Bauleitplanung, aber auch örtliche Pflegeplanung) und Strategien (z.B. Klimaanpasungsstrategie, integrierte kommunale Entwicklungskonzepte) angelehnt. Er hat zum Ziel,

\begin{itemize}
\item gesundheitsbezogene Anliegen, sozialräumliche Besonderheiten und quartiersbezogene Handlungsbedarfe aktuell und vor allem prospektiv (räumlich) darzustellen, 


\item Handlungsempfehlungen und möglichst konkrete Planungsziele und Maßnahmenvorschläge zu erarbeiten, als Handlungskonzept zu formulieren und damit


\item gesundheitsbezogenem, verhältnisorientiertem Handeln in kommunalen Planungen mehr Stringenz, Transparenz, Konsens und vor allem Verbindlichkeit zu verleihen.


\end{itemize}

Um diesem Anspruch gerecht zu werden, ist eine gut aufgestellte GBE zur (integrierten) Bestandanalyse, Bedarfs- und Bedürfnisbestimmung und letztlich als Monitoringinstrument während und nach der Maßnahmenumsetzung extrem hilfreich (vgl. Abb.1). 

\begin{figure}
\scaledgraphics{51b1801d-0dfc-4206-a829-66270db0aee8.jpg}{1}
\caption*{Abbildung 1: Komponenten der Entwicklung eines Fachplans Gesundheit}\label{F19907881}
\end{figure}


Zum Fachplan Gesundheit wurden Vorarbeiten (LIGA.NRW 2011) sowie zwei fiktionale Fachpläne (für den Kreis Gesundbrunnen und die Stadt Healthhausen) publiziert, die auf die jeweiligen Besonderheiten von kreisfreier Stadt- und Landkreisebene eingeht. Diese Dokumente sind abrufbar unter \href{www.lzg.nrw.de/9116816}{LZG.NRW\_Fachplan Gesundheit}. 


In den fiktionalen Fachplänen wurden Aufbau und mögliche Inhalte des Fachplans umrissen (LZG.NRW 2012a und b). Er basiert auf raumbezogenen gesundheitsrelevanten Informationen und kann daraus Voraussetzungen ableiten, die unter anderem die körperliche Aktivität der Bevölkerung oder die Gesundheitsförderung und gesundheitsbezogene Versorgung von vulnerablen Bevölkerungsgruppen verbessern kann. So kann der Plan beispielsweise Bevölkerungsgruppen in mehrfach belasteten Situationen identifizieren und daraus entsprechende Maßnahmenvorschläge ableiten.


Das Konzept zum Fachplan Gesundheit wurde bereits mehrfach erprobt und zeigte vielfältige Potenziale gerade auch im Hinblick auf den Mehrwert integrierter Verfahrensweisen im kommunalen Verwaltungs- und Planungshandeln und eine gesundheitsförderliche Kommunalentwicklung auf. Jedoch wurden auch rechtlich-administrative und ressourcenbezogene Grenzen ersichtlich. So kann ein Fachplanprozess trotz erwarteten Mehrwerts kaum ressourcenneutral angeschoben werden und erfordert ein starkes Engagement auf verwaltungspolitischer Ebene. Ohne Rats- oder Kreistagsbeschluss besteht ein hohes Risiko, dass der Fachplan wirkungslos bleibt (Claßen/Mekel 2018). Dennoch haben zwischenzeitig mehrere Kommunen in NRW begonnen, Fachpläne zu erstellen, u.a. auch im Rahmen der Erarbeitung integrierter kommunaler Präventionskonzepte (Umsetzung Präventionsgesetz). 


Der Satz zu BW war so nicht ganz richtig, der Titel ähnelte sich dort zwar ursprünglich, aber das Konzept hat sich dann doch grundsätzlich unterschieden und mündete – vom Wording her bewusst abgegrenzt – in die Gesundheitsplanung, die dann in BW auch ins neue Gesetz aufgenommen wurde (vgl. Albrich 2017 u. ÖGDG BW). 


\subsection{Gesundheitsorientierte Planung in anderen Planungskontexten}\label{H2296124}



In vielen Fällen steht die GBE, anders als in der zuvor beschriebenen kommunale Gesundheitsplanung, nicht als planungsvorbereitendes Instrument im Mittelpunkt, sie liefert vielmehr Informationen für übergreifende Planungen z.B. im Rahmen der Kommunalentwicklung. Um sich adäquat in diese Planungen einbringen zu können, sollten Gesundheitsberichterstatter\_innen die bestehenden Planungskontexte, -ziele, -strukturen und -ebenen nicht nur im Blick haben, sondern auch entsprechend zu bewerten wissen. 


In diesem Kapitel wird - beispielhaft für den kommunalen Planungshorizont - dargestellt, in welchen Kontexten die GBE hilfreich bis zentral sein kann und welche Bedingungen ideal sind, um die GBE im Kanon unterschiedlichster Planungskontexte zu positionieren und zu stärken.  


\subsubsection{Das Ziel: stets gesundheitsorientiert planen}\label{H7577170}



Im Sinne von Health in All Policies und dem Anspruch einer gesundheitsfördernden Gesamtplanung besteht für den ÖGD immer eine Möglichkeit, sich in kommunale Planungen einzubringen oder diese gar selbst zu gestalten. Dies bedeutet jedoch nicht, dass dies auch genau so geschieht, selbst wenn beispielsweise über ein Landesgesetz zum ÖGD (wie in Nordrhein-Westfalen) diese Option ausdrücklich benannt wird. Die Gründe hierfür sind vielfältig, wie nachfolgend ausgeführt wird. 


\subsubsection{Gesundheit im kommunalen Planungskontext}\label{H7957666}



Zahlreiche kommunale Planungen berücksichtigen aufgrund gesetzlicher Vorgaben und in einer langjährigen Planungstradition stehend bereits vielfältige gesundheitsrelevante Aspekte. Dies betrifft z.B. die Stadtplanung und Stadtentwicklung, die Umweltplanung oder den Immissionsschutz (vgl. \autocite{RodensteinMarianne2012}, \autocite{LandeszentrumGesundheitNordrhein-Westfalen(LZG.NRW)2019} ). Oftmals spiegeln die benannten Aspekte bislang vornehmlich den Gesundheitsschutz wider und dienen der Gefahrenabwehr oder Gefahrenminimierung. Insofern können diese Planungen am ehesten als gesundheitssensible Planung betitelt werden.  


Aktuell vollzieht sich jedoch in der kommunalen Planungspraxis ein grundlegender Wandel (vgl. \autocite[9]{ClaßenThomas2020}). So sind zu den traditionellen Planungsinstrumenten der Stadtplanung und -erneuerung weitere, zumeist integrierte und integrierende Ansätze hinzugetreten. Städtebauförderungsprogramme (zum Beispiel das Programm Soziale Stadt bzw. ab 2020 Sozialer Zusammenhalt) fordern zunehmend auch die Schaffung sozial gerechter und gesundheitsförderlicher Lebensbedingungen, insbesondere in Quartieren. Folglich werden Kooperationen mit anderen kommunal planenden Akteuren, wie zum Beispiel der Sozial- und Bildungsplanung sowie vereinzelt auch der GBE und der Gesundheitsplanung aufgebaut (\autocite{BaumgartSabineundweitere2018}; \autocite{BöhmeChristaundweitere2018}). Häufig stehen diese Kooperationen (siehe auch Kapitel Vernetzung) im Zusammenhang mit einer kommunalpolitischen Programmatik, bei der z.B. eine Gesunde Kommune und die Orientierung an einem Gesundheitszieleprozess explizite Leitbilder sind. In diesen Fällen vollzieht sich der Wandel von einer primär gesundheitssensiblen zu einer gesundheitsförderlich ausgerichteten, gesundheitsorientierten Planung, bei der der GBE eine große Chance zukommt. 


\subsubsection{Was ist gesundheitsorientierte Planung?}\label{H1078715}



Die AG Gesundheitsorientierte Planung des ÖGD in Nordrhein-Westfalen hat 2019 folgende Aspekte identifiziert, die für das Verständnis gesundheitsorientierter Planung von besonderer Relevanz sind (siehe \autocite{esÖffentlichenGesundheitsdienstes(ÖGD)inNordrhein-Westfalen(NRW)}):

\begin{itemize}
\item Kern der gesundheitsorientierten Planung ist die Verhältnisprävention auf Grundlage der Gesundheitsdeterminanten (Verweis Abschnitt Determinanten) mit dem Ziel, positiv auf den Prozess der Lebenszeit in guter Lebensqualität und Gesundheit einzuwirken. Um Gesundheitsrisiken minimieren und Gesundheitsressourcen stärken zu können, bedarf es einer frühzeitigen Einbringung des ÖGD in Planungen.


\item Gesundheitsorientierte Planung braucht ein integriertes Planungsverständnis  im ÖGD.


\item Gesundheitsorientierte Planung ist nicht auf den Wirkraum des ÖGD beschränkt, sondern strahlt in andere Planungsbereiche aus, die einen Einfluss auf die Gesundheit haben: u.a. Stadt- und Raumplanung, Sozialplanung, Jugendhilfeplanung, Umweltplanung, Verkehrsplanung, Pflegeplanung. In diesen Bereichen soll eine gesundheitssensible Planung gestärkt werden. Der Health in all Policies-Ansatz und dessen Steuerungskreislauf sind insofern das zugrunde liegende Leitbild.


\item Die Stadtplanung stellt aufgrund ihrer integrierten Herangehensweise als vermittelnde Planung sowie ihrer Wirksamkeit auf unterschiedlichen räumlichen Ebenen ein wichtiges Bindeglied dar, um gesundheitsorientierte Planung in der Kommune zu realisieren. Denn kommunale Vorhaben der Prävention und Gesundheitsförderung ebenso wie des Gesundheitsschutzes besitzen zumeist einen klaren räumlichen Bezug.


\end{itemize}

\subsubsection{Planungsverständnis im ÖGD im Wandel}\label{H4370740}



Der ÖGD in Deutschland hat sich über lange Zeit selbst im Wesentlichen über die amtsärztlichen Aufgaben der Gesundheitsaufsichtsbehörden (Gesundheitsämter) definiert. Das vorrangige Ziel war, Gesundheitsrisiken zu minimieren und möglichst unbedenkliche Lebensbedingungen zu schaffen. Mit diesem Gesundheitsschutz-Selbstverständnis brachte sich der ÖGD als Akteur auch in Vorhaben der Stadt- und Raumplanung ein (vgl. \autocite{ClaßenThomas2020,RodensteinMarianne2012}). In den vergangenen zwei Jahrzehnten hat sich dieses einseitige Bild des Public Health -Sektors insgesamt und des ÖGD im Speziellen deutlich gewandelt dahingehend, dass gesundheitliche Ressourcen der Bevölkerung stärker berücksichtigt werden müssen. Nunmehr liegt das Ziel - salutogenetisch und durch die Ottawa-Charta der Weltgesundheitsorganisation (WHO) von 1986 motiviert – auch in der Schaffung, dem Erhalt und der Entwicklung gesundheitsförderlicher Lebensbedingungen. Maßnahmen werden demnach dort ergriffen, wo Menschen leben, lernen, arbeiten, sich versorgen etc. Sie folgen damit dem sogenannten „Setting-Ansatz“ und wirken auf unterschiedlichen Ebenen (Lebenswelten): am Lernort, am Arbeitsplatz, in der Wohnung, bei Freizeitaktivitäten, aber auch und insbesondere auf der Regional-, Kommunal- und Nachbarschafts-/Quartiersebene. Das Selbstverständnis als Gesundheitsaufsichtsbehörde wird zunehmend ergänzt um eben diese Setting-bezogenen Ansätze der Verhältnisprävention und Gesundheitsförderung (vgl. u.a. \autocite{BaumgartSabineundweitere2018},\autocite{BöhmeChristaundweitere2018}). 


Bislang waren solche neuartigen Ansätze aufgrund der schwierigen Gesetzgebungskompetenzen im Gesundheitssektor weitestgehend nicht rechtsverbindlich geregelt. Seit Inkrafttreten des Präventionsgesetzes 2015 und mit dem novellierten Leitfaden Prävention des Spitzenverbandes der Gesetzlichen Krankenkassen wird der Kommune insgesamt als auch den kommunalen Quartieren als einem übergreifenden „Setting Kommune“ nunmehr besondere Aufmerksamkeit geschenkt (PrävG 2015, GKV-Spitzenverband 2018). Das ermöglicht dem ÖGD und insbesondere den kommunalen Gesundheitsämtern, das Thema Gesundheit im Sinne von Health in All Policies mancherorts mit eigenen Fördermitteln im Rahmen integrierter kommunaler Entwicklungsstrategien einzubringen und als wichtiger Akteur einer raumwirksamen gesundheitsorientierten Planung wahrgenommen zu werden (vgl. Köckler 2016). Alles in allem lassen sich die aktuellen Entwicklungen auf die verkürzte Formel bringen: Die Stadtplanung nimmt eine ressourcenschonende, gesundheitssensible Stadterneuerung im Bestand in den Blick, während im öffentlichen Gesundheitsdienst die Gesundheitsförderung im Setting Kommune als Aufgabe den Gesundheitsschutz ergänzt. Beides trifft sich auf kleinräumiger (Quartiers-)Ebene und wird idealerweise durch Daten der kommunalen GBE sowie weiterer Quellen integrierter Berichtssysteme unterstützt. 


\subsection{GBE und Gesundheitspolitik}\label{H7792675}



Die Gesundheitsberichterstattung kann ein Instrument zur Unterstützung und Begleitung von Gesundheitspolitik sein. Sie ist aber eine Fachaufgabe. Gesundheitspolitik ist dagegen zum einen durch den Wählerwillen bestimmt, zum anderen muss sie den Ausgleich mit anderen politischen Interessen und Erfordernissen finden. Es heißt oft, Gesundheit sei unser höchstes Gut, aber diese Maxime stößt schnell an die Grenzen der Finanzverteilung zwischen den Ressorts. 


Das Verhältnis zwischen Gesundheitsberichterstattung und Gesundheitspolitik ist daher zwangsläufig komplex. Gesundheitsberichterstattung ist eine Voraussetzung für eine evidenzbasierte Gesundheitspolitik, sie darf aber nicht politische Vorhaben propagandistisch stützen. Damit würde sie ihre Glaubwürdigkeit verlieren und somit letztlich auch ihre Möglichkeiten, über ihre informative Funktion politisch wirksam zu werden.


\subsection{Vorhaben zur Umsetzung des Präventionsgesetzes}\label{H1192227}



Das Präventionsgesetz hat für die Gesundheitsberichterstattung neue Impulse gesetzt. Zwar gibt es eine Verpflichtung zur Berichterstattung nur auf der nationalen Ebene. Die Natiionale Präventionskonferenz muss regelmäßig einen Präventionsbericht für Deutschland vorlegen. Allerdings wird der ÖGD im Präventionsgesetz explizit angesprochen: Die Landesrahmenvereinbarungen zur Umsetzung des Gesetzesauftrags sollen auch regeln, wie der ÖGD einzubinden ist. Prävention auf der regionalen Ebene wird umso bedarfsorientierter, je mehr sie sich auf regionale Daten stützen kann. Hier kommt die Gesundheitsberichterstattung ins Spiel. In praktisch allen 16 Landesrahmenvereinbarungen gibt es einen Bezug auf die Gesundheitsberichterstattung [\textbf{bitte nochmal prüfen}!]. Vereinzelt ist es bereits zur Erstellung eigenständiger Präventionsberichte gekommen. Ein Länder-Arbeitsgruppe hat sich zwischenzeitlich auf ein Set an Indikatoren zur Präventionsberichterstattung verständigt.





\printbibliography[title={Literaturverzeichnis}]
\end{document}
