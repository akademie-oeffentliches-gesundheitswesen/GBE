\documentclass{article}

\usepackage{hyperref}
\usepackage{caption}
                
\usepackage[backend=biber,hyperref=false,citestyle=authoryear,bibstyle=authoryear]{biblatex}
                
\bibliography{bibliography}
            
\usepackage{graphicx}
                
\usepackage{calc}
                
\newlength{\imgwidth}
                
\newcommand\scaledgraphics[2]{%
                
\settowidth{\imgwidth}{\includegraphics{#1}}%
                
\setlength{\imgwidth}{\minof{\imgwidth}{#2\textwidth}}%
                
\includegraphics[width=\imgwidth,height=\textheight,keepaspectratio]{#1}%
                
}
            
\usepackage{tabu}
\begin{document}

\title{GBE02-Wozu GBE?}

\maketitle


Veränderungen im Krankheitsspektrum, demografischer Wandel, Klimawandel und soziale wie gesundheitliche Ungleichheit sind in Deutschland und weltweit gigantische Herausforderungen für die Gesellschaft und Public Health. Verlässliche und gut aufbereitete Gesundheitsinformationen sind die Voraussetzungen für die Entwicklung von Strategien und Konzepten, um auf diese Herausforderungen angemessen zu reagieren, und die Bedingungen für Gesundheit in jeder Alters- und Bevölkerungsgruppe zu fördern und zu stärken. Gesundheitsberichterstattung als Element von Public Health ist dabei das zentrale Instrument für die Bereitstellung von gesundheitsrelevanten Informationen. 


\subsubsection{Anlässe für die Erstellung von Gesundheitsberichten}\label{H6562084}



Vor der Erstellung eines Gesundheitsberichtes ist die Frage, warum und für wen der Bericht erstellt wird, jedes Mal gründlich zu reflektieren. Aus den Antworten leiten sich unter anderem der Umfang des Berichts und der Sprachstil, aber auch die Mitwirkenden am Bericht und insbesondere die Handlungsempfehlungen und die Zeitplanung ab.


Auf den unterschiedlichen administrativen Ebenen (Bund Länder, Kommunen) gibt es stark variierende Anlässe zur Erstellung von Gesundheitsberichten. Anlässe für kommunale Gesundheitsberichte können beispielsweise sein: 

\begin{itemize}
\item \textbf{G}\textbf{rundlage }\textbf{für  Meinungsbildung und Entscheidungsfindung auf der jeweiligen politischen Ebene (z.B. Kreistag oder Stadtverordnetenversammlung) }

Oftmals hat diese Form der Berichterstattung das Ziel Ressourcen zu steuern - zum Beispiel um die Beendigung oder Bewilligung konkreter Maßnahmen, wie beispielsweise Personalstellen oder Sachmittel, voranzutreiben oder Maßnahmen im Sinne des "proportionate universalism" an die kommunalen Bedarfe anzupassen.

\emph{Beispiel: Bewilligung eines Projektes zur Förderung der wohnortnahen sektorenübergreifenden medizinisch-pflegerischen Versorgung}


\end{itemize}
\begin{itemize}
\item \textbf{Grundlage für Meinungsbildung und Entscheidungsfindung auf der Fachebene zur Evidenzbasierung von fachlichen Empfehlungen}



Das Ziel ist es, den eigenen Erfahrungshorizont aus der täglichen Arbeit zu den Bedarfen  um einen Faktencheck zu erweitern

\emph{Beispiel: Handlungsempfehlung für die kommunale Suchtprävention der örtlichen AG Suchtprävention.}


\end{itemize}
\begin{itemize}
\item \textbf{Grundlage für die Festlegung von kommunalen Gesundheitszielen/prioritären Handlungsfeldern}

Ziel ist es die prioritären Handlungsfelder an den kleinräumig unterschiedlichen Bedarfen und Bedürfnissen auszurichten, um für mehr gesundheitliche Chancengerechtigkeit zu sorgen. Für die Verständigung unter den kommunalen Akteuren und die Formulierung gemeinsamer Ziele kann ein Gesundheitsbericht ein geeigneter Weg sein.

\emph{Beispiel: Eine kleinräumige Bedürfnisanalyse zeigt Unterstützungsbedarf für mobilitätseingeschränkte ältere Menschen bei sozialen und gesundheitsfördernden Aktivitäten. Das kommunale Gesundheitsziel: "Die körperliche Aktivität und Teilhabe an der Gesellschaft bei älteren Menschen ist gestärkt" wird festgelegt.}


\item \textbf{Die Zielerreichung eines Gesundheitszieles soll überprüft werden.}

\emph{Beispiel : Die GBE erhält den Auftrag herauszufinden, wie sich die gesundheitliche Lage 10 Jahre nach Einführung der Gesundheitsziele entwickelt hat}



\begin{enumerate}
\item \textbf{Grundlage für die kommunale Beteiligung bei der medizinisch-pflegerischen Versorgungsplanung}

Das Ziel ist, neben der Lage auch die Zuständigkeiten und Gestaltungsmöglichkeiten der kommunalen Akteure transparent zu machen.

\emph{Beispiel: Die hausärztliche Versorgungsstruktur in den Gemeinden und kleineren Städten und die verschiedenen Fördermöglichkeiten und deren Nutzung wird aufgezeigt}

\begin{itemize}
\item Gesicherte und unabhängige Information für die Bevölkerung, die Fachöffentlichkeit und Entscheidungsträger zu gesundheitspolitisch bedeutsamen Entwicklungen. Ziel ist es, den Prozess der demokratischen Willensbildung in der Gesellschaft zu unterstützen.

\emph{Beispiel: Anlassbezogene Berichterstattung zur "Gesundheit bei Asylsuchenden"}





\end{itemize}

\end{enumerate}

\textbf{Wer gibt den Auftrag?}


\end{itemize}

Aus der kurzen und sicher nicht vollständigen Aufzählung wird ersichtlich, dass Gesundheitsberichte aus den verschiedenesten Kontexten heraus entstehen. Dies macht die Erstellung von Gesundheitsberichten spannend, stellt gerade Neulinge im Berufsfeld aber auch vor nicht unerhebliche Herausforderungen. Aufträge einen Gesundheitsbericht zu erstellen können sowohl von der fachlichen Ebene als auch intersektoral veranlasst werden und "intern" oder "extern" vergeben werden. Gerade bei einer "internen" Auftragsvergabe müssen Themen und Berichtsschwerpunkte darüber hinaus gut abgestimmt werden. Dies ist insbesondere dann der Fall, wenn Gesundheitsberichte als Auftrag von "oben" aus der Verwaltungsleitung oder der Politik kommen und zur Umsetzung "intern" an die Fachebene vergeben werden. 


\subsubsection{\textbf{Die GBE im }\textbf{Public Health Action Cycle}}\label{H1901688}



Die Gesundheitsberichterstattung ist eingebettet in den Public Health Action Cycle und dient im Wesentlichen dem Monitoring und der Bedarfsbestimmung (Verweis Abb.). Damit hat sie sowohl eine prospektiv wirkende Funktion im Rahmen der Planung als auch eine retrospektiv  wirkende Funktion bei Kontrolle der Zielerreichung.

\begin{figure}
\scaledgraphics{e44f6caa-bb9c-4108-ba23-5f2688aea1cb.jpg}{1}
\caption*{Stellung der GBE und weiterer Instrumente des ÖGD im Public Health Action Cycle (PHAC)}\label{F52895671}
\end{figure}


Die GBE ist ist idealerweise gemäß der Strategie "Health in All Policies" Grundlage für planerische Prozesse in Städten und Gemeinden, die Auswirkung auf die Gesundheit haben (z.B. Stadtteilentwicklung, ..) 


\subsubsection{\textbf{G}\textbf{BE orientiert sich an Gesundheitsdeterminanten}}\label{H2786889}



Wie dargestellt beschreibt die GBE nicht nur den Gesundheits- und Krankheitszustand der Bevölkerung sowie von unterschiedlichen Bevölkerungsgruppen, sondern zusätzlich verhaltens- und verhältnisbezogene Faktoren, die Auswirkungen für Gesundheit und Wohlbefinden der Bevölkerung haben können (Determinanten für Gesundheit). Das Ziel liegt darin, solche Bedingungen und Strukturen zu identifizieren, die entweder einen großen Impact für die Bevölkerungsgesundheit haben oder mit verhältnismäßig einfachen Mitteln zu verändern sind. GBE - wie auch der Public Health-Bereich insgesamt - orientiert sich am Modell der  den grundlegenden Determinanten von Gesundheit. 


Wie komplex die Wechselwirkungen zwischen Gesundheit und Wohlbefinden und der Lebensumwelt sind, zeigt das „Modell der Gesundheitsdeterminanten im Siedlungsraum“ (im Original Determinants of Health and Well-Being in our Neighbourhood, kurz als Health Map bezeichnet, vgl. \hyperlink{F72875031}{figure 1: Modell der Gesundheitsdeterminanten}) von Barton und Grant (2006). Hier werden die komplexen Beziehungen zwischen den individuellen und sozialen Gesundheitsfaktoren sowie den gesundheitsrelevanten Schlüsselfaktoren von Siedlungsgebieten inklusive ihrer natürlichen und bebauten Umgebung dargestellt.

\begin{figure}
\scaledgraphics{24b771d8-2005-44ea-8d56-c49c62ed732a.jpg}{1}
\caption*{Abbildung 1: Modell der Gesundheitsdeterminanten}\label{F96167441}
\end{figure}


Im Zentrum des Modells befindet sich der Mensch mit seinen individuellen gesundheitsbestimmenden Faktoren wie Alter, Vererbung und Lebensstil. In den Sphären, die das Zentrum umgeben, repräsentieren einzelne Ebenen die unterschiedlichen sozialen, ökologischen, ökonomischen und gesellschaftspolitisch wirksamen Systeme, in die die Menschen eingebettet sind. Das humanökologisch motivierte Modell führt beispielhaft vielseitige Faktoren auf, die sich zunächst wechselseitig beeinflussen und auf den Menschen und seine Gesundheit gesundheitsförderlich, aber auch gesundheitsabträglich, einwirken können. Zudem werden Menschen als Bestandteil des globalen Ökosystems verortet, welches direkt und indirekt auf die Gesundheit einwirkt und umgekehrt. Das gesamte Modell zeigt zugleich, dass zahlreiche Verbindungen zwischen \emph{Healthy Communities} bzw. Nachbarschaften/Kommunen und einer nachhaltigen Entwicklung bestehen. Denn all diese Faktoren sind, dem Ansatz der Nachhaltigkeit folgend („global denken, lokal handeln“), in den globalen Kontext eingebettet (vgl. \autocite{BartonHughundweitere2006}, siehe auch Nachhaltigkeitsziele der Vereinten Nationen (Sustainable Development Goals)).


Bezeichnend ist, dass der Grad der wahrgenommenen persönlichen, verhaltensbezogenen Möglichkeiten, die Determinanten zu beeinflussen, begrenzt ist - im Gegensatz zu gesellschaftlich-politischen Möglichkeiten. Vornehmlich können wir individuell auf die verhaltensorientierten Sphären des Lebensstils und der sozialen Netzwerke einwirken. Doch die Verhältnisse, in denen Menschen aufwachsen und leben, können mittel- und langfristig das Verhalten von Individuen und Bevölkerungsgruppen ebenso wie die Gesundheit direkt nachhaltig beeinflussen (vgl. \autocite{ClaßenThomas2020}). Als Beispiel sei hier der Verkehrslärm genannt. Verhältnispräventive Maßnahmen rücken deshalb heutzutage immer stärker in den Fokus gesundheitsorientierter Interventionen, so auch in der bebauten Umwelt.


Aus Public-Health-Sicht offenbart das Modell allerdings auch einen oft unterschätzten Umstand: die meisten der benannten Faktoren (gerade in den äußeren Sphären) sind zwar grundsätzlich planbar, allerdings befindet sich nur ein Bruchteil im Zugriff des Gesundheitssektors. Schnell wird klar, dass eine gesundheitsorientierte und erst recht gesundheitsförderliche Planung entweder im Rahmen anderer raumbezogener Planungen (zum Beispiel Stadt-, Umwelt-, Verkehrs- oder Sozialplanung) oder idealerweise im Zuge einer integrierten, ressortübergreifenden Planung im Sinne einer gesundheitsförderlichen kommunalen Gesamtpolitik am ehesten nachhaltig umsetzbar ist (vgl. \autocite{LandeszentrumGesundheitNordrhein-Westfalen(LZG.NRW)2019}; \autocite{ClaßenThomas2020}). Das bedeutet im Umkehrschluss für die GBE aber auch, dass diese nicht als isolierte Fachberichterstattung angelegt werden kann, sondern selbst auf Daten aus anderen Berichtssystemen angewiesen ist und stets offen sein muss für Ansätze einer integrierten Berichterstattung. 


\subsection{Selbstverständnis in der GBE}\label{H4406084}



GBE findet mit dem Anspruch statt, handlungsorientiert und planungsrelevant zu sein, d.h. Taten anzustoßen. Sie findet jedoch durch wissenschaftliche Expert\_innen in einer hierarchisch gegliederten Struktur, etwa der Kommune statt und nicht durch diejenigen, welche die Entscheidungen über die Maßnahmenebene treffen. Entscheidungen über folgende Taten können nur von den legitimierten Entscheidungsträger\_innen getroffen werden, seien diese innerhalb der Kommunalverwaltung, in den kommunalpolitischen Gremien oder bei externen Institutionen des Gesundheitswesens oder darüber hinaus angesiedelt. GBE dient der Information und Beratung dieser Entscheidungsträger, sie stellt daher u.a. ein Instrument der Politikberatung dar (Politik in einem weiteren Sinne verstanden, auch Firmen und Institutionen verfolgen eine Politik)\autocite{Brandundweitere2007}.


Um die eigene Rolle als GBE-ler\_in in der Politikberatung zu finden, ist es sinnvoll, das eigene Selbstverständnis im Rahmen dieses Beratungsprozesses zu reflektieren. Drei Modelle und damit verbundene Grundannahmen werden dabei unterschieden\autocite{Brandundweitere2007}:

\begin{enumerate}
\item \textbf{Technokratisches Modell}: Nach diesem Modell folgt die Politik der Wissenschaft und ihren Empfehlungen, es kommt zu einer Verwissenschaftlichung der Politik. Dieses Modell passt zu Prozessen, die vorab weitgehend festgelegt sind, wie dies etwa bei Ausbrüchen von Infektionskrankheiten und den im IfSG hinterlegten Reaktionen der Fall ist. Für die übliche politische Entscheidungsfindung ist das Modell eher ungeeignet, da politische Entscheidungen durch - beispielsweise demokratisch - legitimierte Mandatsträger getroffen werden sollen. GBE-ler\_innen stammen meist aus akademischen Kontexten. Um Enttäuschungen vorzubeugen, gilt es sich zu vergegenwärtigen, dass über die weite Mehrzahl aller Taten nichttechnokratisch entschieden wird.


\item \textbf{Dezisionistisches Modell}: Nach diesem Modell berät eine wertfreie Wissenschaft eine Politik, die auf Basis von Werten und Weltanschauungen Entscheidungen trifft. Das Selbstverständnis des oder der GBE-ler\_in ist es, den Entscheidungsträger\_innen die bestmögliche Informationsbasis für ihre Entscheidungen bereitzustellen. Dadurch sollen evidenz-informierte politische Entscheidungen ermöglicht werden\autocite{Rushmerundweitere2019}. Dieses Modell wird rein formal den meisten politischen Prozessen und Zuständigkeiten gerecht. Es postuliert jedoch eine Wertfreiheit, die in der Gesundheitswissenschaft schwerlich zu finden sein dürfte. Wertorientierungen des oder der GBE-ler\_in etwa im Sinne eines Leitwerts Gesundheit oder der HiAP-Ziele werden im Beratungsprozess nicht ausgeblendet, sondern sind Teil von ihm. 


\item \textbf{Pragmatistisches Modell}: Nach diesem Modell wird eine wertende Wissenschaft postuliert, welche Politik berät und aufgrund der eigenen Wertorientierung gleichzeitig in den Diskurs mit ihr tritt. Entscheidungen werden somit in einem Wechselspiel zwischen Politik und Wissenschaft getroffen. Der oder die GBE-ler\_in wird im Normalfall durchaus für die eigenen Werte 'streiten', wie es in diesem Modell hinterlegt ist. Inwieweit die Entscheidungsprozesse dann eher dezisionistisch oder pragmatistisch stattfinden, hängt von seiner oder ihrer Rolle ab sowie vom Kontext, der ja durch eine Vielzahl weiterer Akteure z.B. aus Kommunalpolitik oder Expert\_innengremien beeinflusst wird (siehe Strukturen ...).


\end{enumerate}

Wichtig für das Selbstverständnis der meist wissenschaftlich geprägten GBE-ler\_innen ist es, die Unterschiedlichkeit der Rationalitäten in Wissenschaft und Politik zu realisieren. Ihre jeweilige Sprache ist auf die unterschiedlichen Adressaten abgestimmt, ihre Planung ist von sehr unterschiedlichen Zeitabläufen bestimmt und sie verfolgen rollengemäß ganz unterschiedliche Ziele (s. Tabelle, nach \autocite{Kurth2006}). Für eine nachhaltig erfolgreiche GBE gilt es, sich an der Schnittstelle zwischen Wissenschaft und Politik mit politischen Rationalitäten vertraut zu machen und diese wenn möglich auch zu berücksichtigen.


\begin{tabu} to \textwidth { |X|X|X| }
\hline



 & Wissenschaftler & Politiker
 \\


Sprache & Fachspezifisch, für Nichtwissenschaftler schwer zu verstehen


 & Oft vereinfachend und populistisch, soll von der ganzen Bevölkerung verstanden werden
 \\


Zeitplanung & Ansammlung von Spezialkenntnissen und Expertise über einen langen Zeitraum & Einhaltung eines Zeitplans geht häufig über Qualität
 \\


Aufmerksamkeitsspanne & Lang: Kumulativer Prozess der Erkenntnisfindung & Kurz: Suche nach schnell verfügbaren Informationen zu einer Vielfalt wechselnder Themen
 \\


Ziele (PPP) & Fortschritt der Wissenschaft, \textbf{P}ublikationen (Impact-Faktor), \textbf{P}atente, \textbf{P}rofessuren & Krisenmanagement, öffentliche Unterstützung, \textbf{P}olitik, \textbf{P}raxis, \textbf{P}opularität
 \\
\hline

\end{tabu}

\subsubsection{\textbf{Ethik und Verantwortung}}\label{H5560507}



Gesundheitsberichterstattung soll handlungsorientiert sein. Damit sind unvermeidlich ethische Fragestellungen verbunden. Zunächst einmal ist weitgehend Konsens, dass an die Gesundheitsberichterstattung die Erwartung an Unabhängigkeit, Sachlichkeit, Überparteilichkeit und Objektivität gestellt wird. Bei der Auswahl der Daten und der Interpretation sind subjektive Einflüsse allerdings inhärent. Eine wissenschaftliche Arbeitsweise, Transparenz und stetige Reflexion über implizite und unerwünschte - da subjektive - Vorannahmen im Team der GBEler sind  Instrumente, um bestmögliche Obejektivität zu wahren.  So kann das derzeit große Vertrauen in eine unabhängige GBE bei Politik, Verwaltung und in der Bevölkerung erhalten bleiben.


In dem Maße, in dem Gesundheitsberichterstattung präventive Programme unterstützt, sollte auch darauf geachtet werden, dass Gesundheitsberichte primär eine informative Funktion haben, weniger eine persuasive. Das unterscheidet Gesundheitsberichte beispielsweise von Broschüren mit Hinweisen zum "richtigen" Gesundheitsverhalten. Inwiefern Gesundheitsberichte also dazu beitragen sollen, direkt das Gesundheitsverhalten der Bevölkerung so zu beeinflussen, wie es gesundheitspolitisch gewollt oder gesundheitswissenschaftlich für richtig gehalten wird, muss kritisch reflektiert werden.


Ein weiterer Aspekt, der hier anzusprechen ist, ist die Ambivalenz des Sichtbarmachens von gesundheitlichen Problemen einerseits und der Gefahr, damit Personen zu stigmatisieren, andererseits. Beispiel: Im Prozess der GBE und dem fertigen GBE-Produkt sollen die Lebenssituationen und unterschiedlichen sozialen Milieus berücksichtigt werden,  ohne dabei zu diskriminieren. Wenn über adipöse Kinder berichtet wird, soll einerseits ein ernstes gesundheitliches Problem zur Sprache gebracht werden, andererseits sollen die betroffenen Kinder nicht auch noch zusätzlich stigmatisiert werden. In gleicher Weise ist darauf zu achten, dass gesundheitliche Probleme etwa bei Menschen mit Migrationshintergrund nicht zur Verstärkung gesellschaftlicher Ausgrenzungstendenzen beitragen.


\textbf{Weiterführende Informationen:}

\begin{itemize}
\item Leitlinie 1 (Ethik) in: Gute Praxis GBE.2.0


\item Kuhn 2016, Gesundheitsberichterstattung, in Schröder-Bäck/Kuhn: Ethik in den Gesundheitswissenschaften, Juventa, S. 384 ff.


\end{itemize}







\subsubsection{}\label{H4753564}



Bezieht sich der Public Action Cycle auf die Gesamtprozess der Gesundheitsplanung im interdisziplinären Prozess ist der Übergang zur Health in all Policies Strategie gelungen. ( siehe Vernetzung)


\printbibliography[title={Literaturverzeichnis}]
\end{document}
