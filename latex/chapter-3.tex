\documentclass{article}

\usepackage{hyperref}
\usepackage{caption}
                
\usepackage[backend=biber,hyperref=false,citestyle=authoryear,bibstyle=authoryear]{biblatex}
                
\bibliography{bibliography}
            
\usepackage{graphicx}
                
\usepackage{calc}
                
\newlength{\imgwidth}
                
\newcommand\scaledgraphics[2]{%
                
\settowidth{\imgwidth}{\includegraphics{#1}}%
                
\setlength{\imgwidth}{\minof{\imgwidth}{#2\textwidth}}%
                
\includegraphics[width=\imgwidth,height=\textheight,keepaspectratio]{#1}%
                
}
            
\begin{document}

\title{GBE03-Strukturen}

\maketitle


Der Öffentliche Gesundheitsdienst ist deutschlandweit auf den Ebenen von Bund, Bundesländern und Kommunen organisiert (siehe Abb. ...), in einigen Bundesländern kommen auf der Ebene zwischen Land und Kommune noch die Regierungspräsidien hinzu. GBE findet v.a. auf den erstgenannten drei Ebenen statt, dies jedoch in unterschiedlicher Intensität. Die Aufgaben und Strukturen des ÖGD sind weitestgehend föderal geregelt, d.h. für die kommunale GBE sind landesgesetzliche Regelungen maßgeblich, die Gesundheitsdienstgesetze der Länder (siehe auch Kapitel 1). Für die kommunale GBE sind die übergeordneten Ebenen von Bund und Land z.B. als Datenhalter, für die Datenaufbereitung und für fachliche Unterstützungsleistungen durch Landesoberbehörden wie z.B. Landesgesundheitsämter von Bedeutung (s.u.).


\begin{center}
\begin{figure}
\scaledgraphics{d56120d9-69fe-4a3b-a811-9306cf7f61c5.png}{0.75}
\label{F93283101}
\end{figure}


\end{center}


\subsection{Kommunale Ebene}\label{H6814286}



Auf der Ebene der Kommunalverwaltung ist der ÖGD zum weit überwiegenden Teil in Landkreisen angesiedelt, z.T. in kreisfreien Städten (o. Stadtkreisen/Städteregionen) und zu einem geringeren Teil übergreifend über mehrere Landkreise und/oder Städte (siehe Abb. ...). Speziell für die sinnvollerweise eng mit der GBE verbundenen Planungsprozesse ist dies bedeutsam, da verschiedene politische Ebenen und Strukturen berücksichtigt werden müssen, um mittels GBE gesundheitspolitische Entscheidungen unterstützen oder aber Gesundheitsbezüge in sonstige politische Entscheidungsprozesse einspeisen zu können (HiAP, siehe ...):


{\raggedleft
\begin{figure}
\scaledgraphics{28019e22-b187-4465-8df9-aa8eebc93ca1.png}{0.75}
\caption*{Aufteilung der kommunalen ÖGD-Einheiten: Der Großteil der kommunalen Gesundheitsämter hat seinen Sitz in Landkreisen, ein kleinerer Teil in Städten, zu kleinem Teil teilen sich mehrere Landkreise u./o. Städte einen kommunalen ÖGD (Quelle: GBE-Monitor, Stand 11-12/2019)}\label{F85841541}
\end{figure}


}

\begin{enumerate}
\item Landkreise: Zu berücksichtigen sind die Landkreisverwaltung sowie die Ebene der Gemeinden und Städte des Kreises.


\item Städte: Zu berücksichtigen ist die Stadtverwaltung sowie ggf. Ortsteilverwaltungen.


\item Kumulierte Organisation (kommunaler ÖGD über mehrere Landkreise u./o. Städte hinweg): Zu berücksichtigen sind die Verwaltungen mehrerer Landkreise und/oder Städte, Gemeindeverwaltungen sowie ggf. Ortsteilverwaltungen.


\end{enumerate}

\begin{center}
\begin{figure}
\scaledgraphics{d6f2d7f1-3336-438f-9216-768a1f38c17e.png}{0.5}
\caption*{Aufteilung der kommunalen ÖGD-Einheiten: ... (Quelle GBE-Monitor, Stand 11-12/2020, Stadtstaaten HH und B unter kreisfreien Städten subsummiert)}\label{F10503981}
\end{figure}


\end{center}


Für die GBE als gesundheitspolitisches Steuerungsinstrument sind diese Gebietskörperschaften relevant, da Vielzahl und Komplexität der zu berücksichtigenden politischen Entscheidungsebenen auch allen für die GBE wesentlichen Prozessen mehr Komplexität verleihen. Gegenüber Städten gilt es schon in Landkreisen deutlich mehr politische Akteure zu berücksichtigen. Der höchste Komplexitätsgrad herrscht, wenn ein ÖGD für mehrere Kreise zuständig ist, da die Kommunalpolitik mehrerer Kreise bzw. Städte und dazu meist eine Vielzahl an Gemeindeverwaltungen zu berücksichtigen sind.  


Innerhalb der Kommunalverwaltung ist die GBE als Aufgabe des ÖGD meist im Gesundheitsamt bzw. Fachbereich Gesundheit o.ä. angesiedelt. Der ÖGD wiederum ist meist einem mehrere Ämter umfassenden Dezernat oder einer Abteilung zugeordnet (siehe Abb. ..., \autocite{SzagunBundweitere2016}). Den unterschiedlichen Dezernaten oder Abteilungen steht die kommunale Spitze vor, d.h. Landrätin oder Landrat bzw. Oberbürgermeister\_in. 


Landkreis oder Stadt sind auch politische Einheiten, kommunalpolitische Entscheidungen z.B. bzgl. Ressourcen werden von den gewählten Vertreter\_innen des Kreistags oder städtischen Parlaments getroffen. Den einzelnen Dezernaten oder Abteilungen sind meist spezifische Unterausschüsse des Kreistags oder Stadtparlaments (Stadtrates) zugeordnet, in denen fachliche Fragen erörtert werden. Auch der ÖGD und die GBE sind daher meist einem solchen politischen Fachausschuss zugeordnet und bringen ihre spezifischen Belange, etwa einen Gesundheitsbericht, dort ein. 


Abhängig von der Gesetzeslage für den ÖGD im jeweiligen Bundesland sind auf kommunaler Ebene teilweise darüber hinaus noch gesundheitsspezifische Planungsgremien angesiedelt (z.B. kommunale Gesundheitskonferenzen). Diese sind häufig direkt dem ÖGD zugeordnet, d.h. ihm obliegt die Geschäftsführung. Diese Gremien sind unterschiedlich zusammengesetzt, Mitglieder sind meist lokale Stakeholder aus den Feldern Gesundheitsversorgung, Gesundheitsförderung und Prävention, mit mehr oder weniger Beteiligung von Patient\_innen und Bürger\_innen. Die Mitglieder sind eine Auswahl von Expert\_innen, sie sind jedoch nicht politisch legitimiert. Daher haben diese Gremien üblicherweise eine beratende Funktion sowie eine eher geringe Ressourcenverantwortung.


Frage: Ist es richtig, dass in einigen KGKen Landräte sitzen?  Ich dachte das sein in BW der Fall. Wie unten ind er Grafik illustriert würde das ja neben der Schnittstelle Soziales/Gesundheit einen weitergehenden HiAP-Ansatz ermöglichen. Abb. angepasst


Antwort: In Hessen gibt es für die kommunale Gesundheitskonferenz keine vorgegebene Struktur- hier ist es  üblich, dass Landrätin/OB die Leitung inne haben, die Gesundheitsämter die Geschäftsführung. Außerdem gibt es in Hessen "regionale Gesundheitskonferenzen, die gesetzlich geregelt sind: Durchschnittlich 4  Gebietskörperschaften sollen auf den Gebieten Prävention und Versorgung Lösungen erarbeiten - Abb wurde angepasst, Punkt sollte im Kapitel Planung noch aufgenommen werden


die Abbildung ist nur ein Beispiel-trifft für Hessen nicht zu: Pfeil müßte von Kommunaler Spitze + ÖGD zu KGK gehen wurde mit LR angepasst

\begin{figure}
\scaledgraphics{dc0d4f16-5397-4981-a73c-357b6adf9a16.png}{1}
\caption*{Rahmenbedingungen und Fragen für eine erfolgreiche kommunale Koordinationsaufgabe des ÖGD, dargestellt an einer kommunalen Gesundheitskonferenz (KGK) in einer beispielhaften Verwaltungsstruktur A: Zuständigkeit, Zusammensetzung und Budget der KGK, Mandat und Kooperation gegenüber welchen kommunalpolitischen Ausschüssen, Form und Ausmaß partizipativer Planungsprozesse. B: Einordnung des ÖGD in die kommunale Verwaltung (z.B. Sozialdezernat, Sozialbürgermeister\_in), eng davon abhängend Qualität der Schnittstellen zu anderen Dezernaten und Ämtern sowie Möglichkeiten für eine integrierte Sozial- und Gesundheitsberichterstattung. C: Interne ÖGD-Struktur, personelle Ausstattung und hierarchische Zuordnung von GBE und Koordination inkl. struktureller Qualität der Kooperation}\label{F42572711}
\end{figure}


Das alles klingt nicht nur komplex, es ist es auch. Für die kommunale GBE sind die kommunalpolitischen Strukturen jedoch in vieler Hinsicht bedeutsam\autocite{AlbrichC2017}. Will GBE 'Daten für Taten' kommunizieren, d.h. planungsrelevant sein, findet diese Planung in einem oder mehreren der o.g. Gremien statt. Am Ende gilt es meist, Entscheidungen über Ressourcen zu beeinflussen, d.h. die dafür zuständigen Entscheidungsträger zu überzeugen. Die Reichweite von gelingender GBE geht somit immer 'über das Amt hinaus' und alle relevanten Strukturen 'über das Amt hinaus' sollten von Anfang an mitgedacht werden, damit GBE gelingt. Für die GBE zu berücksichtigende Aspekte ergeben sich aus den Hierarchieebenen, der kommunalpolitischen Ausschüssen und Expert\_innengremien:

\begin{itemize}
\item Hierarchieebenen: Ein Gesundheitsbericht ist immer eine offizielle Verlautbarung der Kommunalbehörde, die auch von der kommunalen Spitze genehmigt sein muss. Die für die GBE zuständige Person ist innerhalb des kommunalen ÖGD meist auf der zweiten oder dritten Hierarchieebene angesiedelt. Über Amtsärztin oder Amtsarzt liegen ein bis zwei weitere Hierarchieebenen, sodass ein Bericht nicht selten über vier Hierarchieebenen hinweg genehmigt werden muss, bevor er nach außen gehen kann. Das kostet nicht nur Zeit, es kann auch zu Konflikten führen, sofern nicht eine gewisse Vorabstimmung zwischen den relevanten Ebenen stattgefunden hat (v.a. bzgl. Handlungsempfehlungen, siehe Abschnitt Vernetzung). Ein Teil dieser Vorabstimmung obliegt meist der Amtsleitung, da diese einen direkten Zugang zu den übergeordneten Führungskräften hat. Günstig ist es daher, wenn zumindest im ÖGD selbst so wenige Ebenen wie möglich zwischen GBE und Amtsleitung liegen (siehe Empfehlungen zur Organisation ...).


\item Kommunalpolitische Gesundheitsausschüsse: : Kommunalpolitische Planungen und ressourcenrelevante Entscheidungen zur gesundheitsrelevanten Themen werden hier erörtert und vorbereitet. Die Entscheidungen werden vom Kreistag bzw Stadtparlament getroffen. Die Ausschüsse sind von Kommunalpolitiker\_innen verschiedener Fraktionen besetzt, die informiert bzw. evtl. überzeugt werden müssen. Häufig ist dem ÖGD-Dezernat bzw. der ÖGD-Abteilung ein spezifischer Unterausschuss von Kreistag oder Stadtparlament zugeordnet, z.B. einem Sozialdezernat ein 'Ausschuss für Soziales und Gesundheit'. Abhängig von der Dezernatszuordnung sind dann auch die 'üblichen' Themen und die Expertise, die in einem solchen Ausschuss angesiedelt sind. Für die GBE gilt es dies zu berücksichtigen, d.h. soweit möglich zu antizipieren, welche Themen dort auf welche Weise diskutiert werden dürften. Es kann sinnvoll sein, Vertreter\_innen der politischen Fraktionen etwa über die Einbindung in die Gesundheitskonferenz frühzeitig in die Berichtsthematik zu involvieren, um eine konstruktive Diskussion vorzubereiten. Das Gesundheitsamt kann als eigenständige  Fachabteilung/Fachbereich der Landrätin/dem Landrat direkt unterstellt sein, meistens ist es jedoch dem Dezernat eines Beigeordneten bzw. Dezernenten zugeordnet. Die häufigsten Zuordnungen des ÖGD sind bei Dezernaten für Soziales und/oder Jugend sowie Ordnung und/oder Veterinärwesen (GBE-Monitor 11-12/2019). Die Dezernatszuordnung spielt neben der kommunalpolitischen Bedeutung auch eine Rolle dafür, wie einfach oder kompliziert es sich gestaltet, mit anderen Ämtern und Fachbereichen zu kooperieren und damit eine Integration von GBE und Planung zu realisieren (siehe Integrierte GBE ...). Eine direkte Unterstellung des Fachbereichs Gesundheit unterhalb der kommunalen Spitze ermöglicht es andererseits, der Gesundheitplanung einen hohen und eigenständigen Stellwert zu verleihen.Expert\_innengremien zur Gesundheitsplanung: Die spezifischen und meist nur beratenden Planungsgremien wie Gesundheitkonferenzen sind häufig in mehrfacher Hinsicht relevant für die GBE. Aus ihnen kommen nicht selten Anregungen oder Aufträge für Berichtsthemen und fast immer stellen sie einen wesentlichen Adressaten der GBE statt. Auch wenn Gesundheitskonferenzen nicht demokratisch legitimiert sind, ist ihr Votum als beratendes Expert\_innengremium wesentlich dafür, ob die GBE zu Taten führt. Es gilt daher, die Diskussion in diesen Gremien vorzubereiten bzw. Mitglieder der Gremien schon vorab sowie in den Prozess einzubeziehen (siehe Vernetzung ...). Essentiell für die notwendigen Abstimmungsprozesse sind häufig auch die Schnittstellen zwischen den Expert\_innengremien und o.g. politischen Gremien. Gute Voraussetzungen lassen sich beispielsweise gewährleisten, indem Mitglieder aller wichtigen politischen Fraktionen auch an der Gesundheitskonferenz beteiligt sind. Sofern in der Gesundheitskonferenz aktive Akteur\_innen wie beispielsweise Krankenkassen oder Ärzt\_innenschaft direkt für ein Planungsfeld zuständig sind, können auch in der Konferenz selbst weitreichende und ressourcenrelevante Entscheidungen getroffen werden. Voraussetzung dafür ist jedoch, dass die Konferenz gut etabliert ist. Eine wichtige Rolle spielt dafür häufig auch die Offensichtlichkeit einer Bedarfslage, wie sie sich z.B. aus der GBE ergeben kann.


\end{itemize}
\begin{itemize}
\item \subsubsection{GBE-Struktur innerhalb des kommunalen ÖGD}\label{H718856}



\item Grundsätzlich reichen Steuerungsaufgaben wie GBE und Planung immer über das Gesundheitsamt hinaus. Die Amtsleitung ist stets mehr oder weniger in sie eingebunden, sei es durch die entstehende Öffentlichkeit, die Diskussionen in kommunalpolitischen Gremien oder in kommunalen Gesundheitskonferenzen. Das macht einerseits eine Unterstützung dieser Aufgaben durch die Amtsleitung zur unabdingbaren Voraussetzung, andererseits sollte die Ansiedlung dieser Aufgabenbereiche die notwendige Interaktion ermöglichen und befördern. Sinnvoll ist - abhängig von der Größe des Aufgabengebiets - die Installation als Abteilung, Sachgebiet oder Stabsstelle direkt unterhalb des Leitungsebene. Die Arbeitsfelder von Planung und GBE sollten möglichst in einer gemeinsamen Struktur integriert sein. Falls Gesundheitsförderung und Prävention prioritäre Planungsthemen darstellen, sind integrierte Arbeitseinheiten für GBE, Planung, Prävention und Gesundheitsförderung eine sinnvolle Lösung, die sich vielerorts bewährt hat.


\end{itemize}

\subsection{Landesebene}\label{H465502}



Gesundheitsberichte auf Landesebene werden entweder von den Gesundheitsministerien (z.B. in Berlin, Hamburg oder Schleswig-Holstein) oder oberen Landesgesundheitsbehörden (Landesgesundheitsämtern) erstellt. Sie haben die Situation im Land insgesamt im Blick, berücksichtigen dabei aber häufig auch regionale Unterschiede und berichten dazu auch Daten auf Kreisebene. Für die kommunale Gesundheitsberichterstattung kann das als Datenquelle oder Vergleichsreferenz dienen. Des Weiteren unterstützt die Landesebene die kommunale Gesundheitsberichterstattung z.B. durch Fortbildungen/Tagungen oder die Erstellung von Handlungshilfen (https://www.lgl.bayern.de/gesundheit/gesundheitsberichterstattung/methoden\_handlungshilfen/index.htm), teilweise auch durch die Bereitstellung von Berichtevorlagen. So gibt es beispielsweise in Bayern eine "GBE-Berichtsschablone", einen vom Layout her vorkonfektionierten kurzen Übersichtsbericht mit 18 Indikatoren, in den nur noch die kommunalen Daten eingetragen werden müssen.


Die Länder stimmen sich untereinander in der Arbeitsgruppe Prävention, Gesundheitsberichterstattung, Rehabilitation und Sozialmedizin der Obersten Landesgesundheitsbehörden (AOLG) ab.  Auf dieser Ebene wurde 2003 auch ein gemeinsamer Gesundheitsindikatorensatz der Länder vereinbart, um zumindest einige Gesundheitsindikatoren ländervergleichbar zu haben. Das Robert Koch-Institut, das Statistische Bundesamt und die Akademie für Öffentliches Gesundheitswesen sind ständige Gäste dieser Arbeitsgruppe. Ein regelmäßiger Bund-Länder-Workshop zur Gesundheitsberichterstattung, den das Robert Koch-Institut in Abstimmung mit den Ländern veranstaltet, stellt auch die vertikale Verständigung zwischen Bund und Ländern sicher.





\subsection{Bundesebene}\label{H5727488}



Auf der Ebene des Bundes ist das Robert Koch-Institut zusammen mit dem Statistischen Bundesamt für die Gesundheitsberichterstattung zuständig. Das Robert Koch-Institut verantwortet die inhaltliche und konzeptionelle Ausgestaltung und Weiterentwicklung des Berichtswesens sowie die die GBE-Publikationen. Das Statistische Bundesamt stellt die wesentliche Datenquellen im \href{https://www.gbe-bund.de}{Informationssystem der Gesundheitsberichterstattung des Bundes (IS-GBE)} bereit\autocite{ThomasZiese2000}.


Im IS-GBE stehen Informationen und Daten aus über 100 Datenquellen zur Verfügung. Viele Tabellen sind entsprechend der eigenen Fragestellung modifizierbar, auch Grafiken können erstellt werden. Das Angebot wird kontinierlich ergänzt und aktualisiert. Im Rahmen des Gesundheitsmonitorings führt das RKI in regelmäßigen Abständen bundesweite repräsentative Befragungs- und Untersuchungssurveys durch (siehe Datenquellen)


Akteurinnen und Akteure im Gesundheitswesen,  der Gesundheitswissenschaft sowie Datenhalter und internationale Expert\_innen werden über die Kommission „Gesundheitsberichterstattung und Gesundheitsmonitoring“ am Robert Koch-Institut in die Arbeit der Gesundheitsberichterstattung des Bundes einbezogen. Die Berichterstatter\_innen des RKI arbeiten in zahlreichen gesundheitspolitischen und wissenschaftlichen Gremien und Prozessen auf nationaler sowie internationaler Ebene mit, sodass auch auf diesem Wege ein ständiger Austausch mit der Politik, Forschung und Praxis sichergestellt wird\autocite{LampertThomasundweitere2010}.


\printbibliography[title={Literaturverzeichnis}]
\end{document}
