\documentclass{article}

\begin{document}

\title{GBE05-Infobox Datenschutz}

\maketitle








In der Gesundheitsberichterstattung wird häufig mit aggregierten Sekundärdaten gearbeitet. In diesem Fall spielt der Datenschutz meist keine Rolle. Der Datenschutz kommt immer dann ins Spiel, wenn personenbezogene Daten verarbeitet werden, die Daten Rückschlüsse auf Personen zulassen oder schutzbedürftige Interessen anderer Art (z.B. von einzelnen Ärzt\_innen) tangiert werden. Dann kann Datenschutz ein heikles Thema werden, dessen Brisanz in der Gesundheitsberichterstattung gerne unterschätzt wird. Grundsätzlich keine Probleme mit dem Datenschutz gibt es, wenn die (schriftliche) Einwilligung der Personen vorliegt, deren Daten verarbeitet werden sollen. Ansonsten gilt das Prinzip, dass Daten nur für den Zweck verarbeitet werden dürfen, für den sie erhoben worden sind, es sei denn, es gibt eine gesetzliche Grundlage für eine Nutzung dieser Daten in anderen Zusammenhängen wie z.B. im Rahmen der Gesundheitsberichterstattung.


Gesundheitsdaten sind besonders sensible Daten und daher in besonderer Weise schützenswert; das heißt, in einem Gesundheitsbericht dürfen unter keinen Umständen Einzelpersonen gegen ihren Willen identifizierbar sein. Die gängige Strategie, Rückschlüsse auf Einzelpersonen zu verhindern, ist das Aggregieren (Zusammenfassen) von Daten. 


In der kleinräumigen Berichterstattung kommt man schnell mit dem Datenschutz in Konflikt. Selbst bei Erhebungen mit großen Fallzahlen wie der jährlichen Schuleingangsuntersuchung ergeben sich durch eine kleinräumige Darstellung von Indikatoren wie Übergewicht/Adipositas häufig nur geringe Fallzahlen. Hier  ist das Aggregieren von Daten aus mehreren Erhebungsjahren sinnvoll, damit sich die Fallzahl bei gleichbleibender räumlicher Bezugsebene vergrößert. Dabei besteht allerdings die Gefahr, dass durch die Zusammenfassung der Daten aus mehreren Jahren zeitliche Veränderungen verdeckt werden, 


Eine andere Möglichkeit ist das Zusammenfassen von Raumeinheiten, entweder nach administrativen Kriterien (Stadtbezirke anstellen von Stadtteilen), was jedoch die interne Heterogenität erhöht, oder auf der Grundlage inhaltlicher Überlegungen, indem man sozialstrukturell oder demografisch ähnliche Gebiete in unterschiedliche Gebietstypen zusammenführt.


Zu stark differenzierende Analysen führen tendenziell zu geringen Fallzahlen, die datenschutzrechtlich bedenklich sein können, wie eine geringe Besetzung von Tabellenzellen. Es gibt unterschiedliche Vorgaben, ab wann eine Fallzahl genügend groß ist..


Im Zweifelsfall ist die Rücksprache mit dem Datenschutzbeauftragten des Landratsamtes bzw. der Stadtverwaltung notwendig. Je nach Situation kann eine datenschutzrechtliche Prüfung sehr zeitintensiv sein, so dass dieser Schritt angemessen in der Projektplanung zu berücksichtigen ist. 


Die kleinräumige Berichterstattung kann auch durch Geheimhaltungsvorschriften im Bereich der amtlichen Statistik betroffen sein. Derzeit werden z.B. in der Todesursachenstatistik bei Todesursachen mit kleiner Fallzahl die Daten von den Statistischen Ämtern nur eingeschränkt bereitgestellt. So werden etwa Suizide auf Kreisebene nicht vollständig ausgewiesen. Dies betrifft auch die Raten der Suizide insgesamt auf Kreisebene, obwohl sie keinen Rückschluss auf Einzelpersonen zulassen.













\end{document}
