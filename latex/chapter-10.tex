\documentclass{article}

\usepackage{tabu}
\begin{document}

\title{GBE10 Ideenspeicher FEHLER + Stellenbeschreibung}

\maketitle


How to mess with GBE…




\begin{itemize}
\item Ich will einen Bericht schreiben, weiß aber nicht warum


\item Ich schreibe einen Bericht ganz alleine


\item Wenn ich etwas nicht weiß, frag ich niemand - soll ja keiner merken


\item Ich habe (langwierige) Abstimmungsprozesse nicht eingeplant


\item Ich habe mich nicht mit der Amtsleitung abgestimmt


\item Ich habe meine Handlungsempfehlungen nicht abgestimmt


\item Ich fang mal einfach so mit einem Bericht an


\item Ich glaube, dass die Zahlen einfach so die Wahrheit sagen.


\item Ach Morbidität und Mortalität sind unterschiedliche Statistiken?


\item Das sind doch meine Daten, wieso wollen denn jetzt andere die Daten interpretieren?


\item Der Bericht ist ein dickes Buch, das keiner zu lesen scheint


\item zu schnell zu viel zu wollen!


\item nach dem ersten Rückschlag aufgeben. Man braucht einen langen Atem!


\item Ohne Konzept starten und einfach über alle Indikatoren z. B. aus den SEU berichten


\end{itemize}

\subsection{Ausschreibung einer GBE-Stelle}\label{H6109369}



\textbf{Heiß oder kalt? Beides!}


\textbf{Das Gesundheitsamt "Im siebten Himmel" sucht für seine Gesundheitsberichterstattung einen Menschen, der Spaß daran hat, Gegensätze zu leben:}


Sie sind integrativ, kommunikativ und gehen gern auf andere Menschen zu; gleichzeitig lieben Sie das einsame Geschäft der Datenauswertung.


Sie sind leidenschaftliche\_r Expert\_in Ihres eigenen Fachgebietes; gleichzeitig lassen Sie sich gern auf interdisziplinäres Arbeiten ein.


Sie arbeiten nach wissenschaftlichen Kriterien; dennoch finden Sie eine angemessene Sprache für unterschiedlichste Adressaten.


Sie reagieren schnell; gleichzeitig haben Sie einen langen Atem bei Projekten, die auf Jahre angelegt sind.


Sie arbeiten strukturiert, aber trotzdem kreativ. 


Sie haben keine Scheu vor Zahlen und großen Datenmengen, aber Sie sind kein Nerd.


Sie arbeiten selbstständig und zielbezogen, Ihre Partner\_innen verlieren Sie dennoch nicht aus dem Blick.


Obwohl Sie in einer hierarchischen Struktur landen, leben Sie partnerschaftliches, motivierendes Arbeiten.


Ihr Blick ist auf das große Ganze gerichtet, aber die Details vergessen Sie nicht. ALTERNATIV: Ihr Blick ist auf die Gesamtgesellschaft gerichtet, aber benachteiligte Menschen (sozial, gesundheitliche, strukturell) stehen im Mittelpunkt.


Und noch ein unverzichtbares Merkmal:


Der Satz "Gesund ist wer nicht krank ist" lässt Ihnen die Haare zu Berge stehen.





\begin{tabu} to \textwidth { |X| }
\hline




 \\



 \\



 \\



 \\



 \\



 \\



 \\



 \\



 \\
\hline

\end{tabu}



\end{document}
